\documentclass{article}
\usepackage{hyperref}
\usepackage{float}
\usepackage{listings}
\begin{document}
\title{Metrics of successful websites and companies}
\author{Danai Avratoglou}
\date{January 2017}
\maketitle
\pagebreak 
\tableofcontents
\pagebreak  
\section{Introduction}
 
An on-line presence of a company was not an important factor of its success until a few years ago. Taking although into account the vast spread of the impact that internet has on consumers regarding their choices this hypothesis is not valid any more. Companies are obliged by the trends to be active on-line and to maintain a website that depicts the image they want their consumers to perceive. The purpose of this paper is to understand the relationship that exists between the website of a company and its success. Trying to comprehend this relationship a comparison will take place between specific metrics of the websites of the Fortune's 500 more successful companies  and their financial status. By performing regression models and statistical analysis this paper will try to explore which metrics of a company's website influence its success.   
 
\pagebreak 
\section{Data gathering}
The first step in order to contact this research is to conclude to the companies that are going to be examined. Since the purpose of this paper is to see if the website metrics that will be examined are influencing the success of the company it is a good idea to examine websites of some already successful companies and try to find out what they have in common.

\subsection{Data Source - Fortune 500}
The Fortune 500 is an annual list compiled and published by Fortune magazine that ranks 500 of the largest U.S. corporations by total revenue for their respective fiscal years. The list includes public companies, along with privately held companies for which revenues are publicly available.\cite{key1, key2}\\ 
For the purposes of this paper we will use this list of companies and we will examine their websites in order to understand if they indeed have something in common or if their success is irrelevant with their on-line presence.\\
The first thing that we will need is a list of the names that are include in the fortune 500. The easiest way to obtain this list is from the following article:
\href{url}{http://www.zyxware.com/articles/4344/list-of-fortune-500-companies-and-their-websites}.\ The way that we will obtain the list will be explained in the 2.4 Scripts section of this chapter.
In the following table one can see the 50 most succesful companies that are included in the Fortune 500 during the period this paper is taking place. The rest of the list is available in the Appendix A.\ref{App:AppendixA}
\begin{table}[H]
\centering
\caption{Fortune 500 - 50 first companies}
\begin{tabular}{lll}
\hline
 \\ 1. Walmart 
&  2. Exxon Mobil 
&  3. Apple 
\\ 4. Berkshire Hathaway 
&  5. McKesson 
&  6. UnitedHealth Group 
\\ 7. CVS Health 
&  8. General Motors 
&  9. Ford Motor 
\\ 10. AT\&T 
&  11. General Electric 
&  12. AmerisourceBergen 
\\ 13. Verizon 
&  14. Chevron 
&  15. Costco 
\\ 16. Fannie Mae 
&  17. Kroger 
&  18. Amazon.com 
\\ 19. Walgreens Boots Alliance 
&  20. HP 
&  21. Cardinal Health 
\\ 22. Express Scripts Holding 
&  23. J.P. Morgan Chase 
&  24. Boeing 
\\ 25. Microsoft 
&  26. Bank of America Corp. 
&  27. Wells Fargo 
\\ 28. Home Depot 
&  29. Citigroup 
&  30. Phillips 66 
\\ 31. IBM 
&  32. Valero Energy 
&  33. Anthem 
\\ 34. Procter \& Gamble 
&  35. State Farm Insurance Cos. 
&  36. Alphabet 
\\ 37. Comcast 
&  38. Target 
&  39. Johnson \& Johnson 
\\ 40. MetLife 
&  41. Archer Daniels Midland 
&  42. Marathon Petroleum 
\\ 43. Freddie Mac 
&  44. PepsiCo 
&  45. United Technologies 
\\ 46. Aetna 
&  47. Lowe's 
&  48. UPS 
\\ 49. AIG 
&  50. Prudential Financial 
&
 \\ \hline

\end{tabular}
\end{table}

\subsection{Metrics}
Now that we have declared the companies that we are going to use we will also need to decide which metrics are we going to examine for each site. Since we cannot have access to metrics such as traffic we will have to examine metrics that are more related to how the site is structure and what exactly does the initial page of each site includes. Initially we can divide the metrics in two major categories:
\begin{itemize}
\item What we see?
\item What lays behind of what we see?
\end{itemize}
In the first category we are referring to metrics that can easily be conceived by the naked eye as well. For example the images that a website is using in its landing page. How many there are and if they are big or small.\\
The second category is not so obvious and it includes informations that usually is visible only to the web developer or the creator of the page. The information here are being given from the html code of a site. For instance we can see if the html code has any errors, that can lead to some malfunction in the site for example.\\
Now that we have a first understanding of the two main categories that the metrics we will use are divided in, we can see in detail the metrics that will be examined in this paper:
\paragraph{Loading time:}One aspect of a website that is crucial is the time it takes for it to load. Nowadays that the internet speed is going higher and higher most people do not have the patient to wait for a page to load. That is why we think that a metrics that should be definitely included in this research in the loading time of the initial page of a site.
\paragraph{Number of links:}When someone is browsing through the internet in many cases they are not completely sure what exactly they are looking for and that is why in a website it would be wise to have some links that can direct the user to find what he wants.These links can either direct the user in another page of the same site, in which case the link will be characterized as an internal link. Or they can lead the user to another site, where in that case the link is characterized as an external link.For the purpose of this paper since it is not so clear which type of links are more important to a user we will examine both the internal and the external links.
\paragraph{Social media:}Our era is marked by the social media wave that has changed our lifestyle and our daily habits. So it would be considered an overview if we didn't take under consideration the number of social media that the company chooses to participate in. Even though they can also be considered as external links of the website we will examine them separately in order to see if any particular social medium effects the company's revenues.
\paragraph{Number and size of images:}Since the site is the first thing that a user will see for the company and there is a famous quote that says that \textit{"First impressions counts"} we should also examine how the companies decide to visualize their landing page. In other words to see how many images they include in it and more on that what sizes are those images.\\ It is completely different to see only one huge image in the landing page of a site with not many words or descriptions than to see many small images with different information. The purpose is to examine if these type of diversities between the examined websites are actually related to how they are doing profit wise.
\paragraph{Type of images:} This kind of information is a little more complicated for a simple user to understand, but in many cases it plays a very important role. For example some websites are using specific type of images or banners that are not compatible with all the browsers, leading the user to see some break points in the website and even stop visiting it. That is why we believe is important to review this metric as well.
\paragraph{Number of words:} They say a picture worth a thousand words but that is not enough in our case. After exploring the number, sizes and type of pictures that a website is using we should also explore the number of words it is using to accompany the images and complete the outcome that a user will come across. The metrics we will use will be two. The first one will be the total words that are being used in the landing page and the other one will be the total unique words that are being used. When we are referring to unique words we mean words that are not so commonly use such as "a" or "and" and they give an air of individuality to the text. By using this metric we would try to see if the words that are being used are just as important as the actual content and if the words can make a difference.
\paragraph{Readability index:}In order for the previous metrics to be completed we will also have to take under consideration how comprehending is the text used in the websites for the users. This information can be obtained by calculating the readability index of the website.How the readability index is being calculating will be furthered explained in another section.
\paragraph{HTML Validation:} Moreover we will have to check the quality of the html code behind the website we are seeing. Are there any mistakes in the code for example any brackets that opened and never closed or any links that do not work. We will examine again two different metrics here the number of errors and the number of warnings. The warning are parts of the code that even though they work at the time there is a good chance to malfunction if any changes or addition are to be made to the html code. 
\subsection{Python}
After having a first look into the variables/ metrics we will use in order to contact this research we should also see how we are going to obtain all this information.\\
Since all of this information can be subtract from the html code of a company's website we should use a programming language in order to download the html pages and then to extract the specific metrics we want to examine from them.\\
For the purposes of this paper the programming language that will be used for downloading the metrics from the websites is Python. More specifically the version of Python that will be used is the 2.7 one. Python is a widely used high-level programming language used for general-purpose programming, created by Guido van Rossum and first released in 1991. An interpreted language, Python has a design philosophy which emphasizes code readability (notably using white space indentation to delimit code blocks rather than curly braces or keywords), and a syntax which allows programmers to express concepts in fewer lines of code than possible in languages such as C++ or Java. \\
The language provides constructs intended to enable writing clear programs on both a small and large scale. Furthermore the way that Python allows a user to programming is common to all users which gives this language a leverage as a program build in Python can be easily understood from another user without any difficulty.\\
The environment that is going to be used is from the Anaconda package which is a free open source distribution of the Python and R programming languages for large-scale data processing, predictive analytics, and scientific computing, that aims to simplify package management and deployment. To be more precise from this package we are going to use the Jupyter Notebook.\cite{key4} \\
\subsection{Scripts}
In order to gather all the needed metrics we had to create many small scripts so as to collect them. In this section we will present in detail the code that was used in order to extract the information that later will help us contact the analysis of the relationship between those metrics and the company's status.
\subsubsection{Fortune data}
The first step is to download and gather the names of the companies that we are going to examine and also the url of their websites and finally their ranking.Those information can be found in the following url as he have already mentioned in the beginning.\href{url}{http://www.zyxware.com/articles/4344/list-of-fortune-500-companies-and-their-websites}.\\
The way to keep only the those three informations as separate variables is by separating from the html code of this page the needed information.\\
The first step is to create 3 empty lists where we will include the informations we are going to extract. The first list will contain the rank of each site, the second one will contain the name of the company and the 3rd one will contain the actual link of the company's site:
\begin{lstlisting}[language=Python]
list_company_number =[]
list_company_name = []
list_company_website = []
\end{lstlisting}
The second step is to upload some libraries that will help us create this function but also the ones that are going to follow.
\begin{lstlisting}[language=Python]
import urllib
import urllib2
import time
import os
from bs4 import BeautifulSoup
import re
import numpy as np
import pandas as pd
import matplotlib.pyplot as plt
\end{lstlisting}
Finally the third step is to create the function that will firstly download the html code of the url at hand, secondly keep only the part of the code that we need to examine and thirdly save this part into the empty lists we created above. This function is called websites and takes as variable to work only the url of the site we need to examine: 
\begin{lstlisting}[language=Python]
def websites (url): 
    from time import time
    start = time ()
    browser = urllib2.build_opener() 
    browser.addheaders = [('User-agent', 'Mozilla/5.0')]
    response = browser.open(url)
    myHTML = response.read()
    soup = BeautifulSoup(myHTML,"lxml")    
    o = 0
    td_list =[]
    for row2 in soup.html.body.findAll('td'):
        td_list.insert(o, row2)
        o = o + 1
    a = 0
    b = 1
    c = 2
    list_numbering = 0
    for i in range (0,500):        
        num = str(td_list[a])
        company = str(td_list[b])
        site = str(td_list[c])
        c_num = re.findall('>(.+?)</td>',num)  
        c_num = str(c_num[0])
        c_name = re.findall('>(.+?)</td>',company)
        c_name = str(c_name[0])
        c_site = re.findall('">(.+?)</a>',site)
        c_site = str(c_site[0])        
        list_company_number.insert(list_numbering,c_num)
        list_company_name.insert(list_numbering,c_name)
        list_company_website.insert(list_numbering,c_site)
        a = a + 3
        b = b + 3
        c = c + 3
        list_numbering =  list_numbering + 1 
    end = time ()
    duration = round (end - start, 1)
    minutes = round (duration /60, 1)
    print 'The lists are ready in ', duration, ' seconds'
    print 'The lists are ready in ', minutes, ' minutes'
\end{lstlisting}
The steps we followed to create the following function are the following:
\paragraph{Step 1} We create a fake browser that we are going to use in order to open the page and downloaded. The reason we do that is that many sites do not allow us to download their page because they are afraid of stealing important information. Since we are not using any private information we use this method to avoid issues while trying to open the html page at hand.    
\paragraph{Step 2} We open the url and we read it while saving it in the variable "myHTML".
\paragraph{Step 3} With the help of the BeautifulSoup library we read the page as a lxml file and then for each row of this file we are looking for the "td" parts of the code where the informations we want are included.
\paragraph{Step 4} Since we need the names and the urls of all the 500 sites we created a loop from 0 to 500 where for each i we try to isolate the part of the code that contains the information that we want. Moreover even thought it seems that with this loop we calculate 501 numbers since in Python the second bracket is always open we actually count from zero to 499.
\paragraph{Step 5} We use reg expressions\footnote{A regular expression is a special sequence of characters that helps you match or find other strings or sets of strings, using a specialized syntax held in a pattern.} in order to state precisely what part of the already selected code we want to keep.
\paragraph{Step 6} We insert with a specific order the names, the ranking and the url to the corresponding lists and finally we create a text that will appear when the function is completed. Here we have also calculated the time that this function took to be completed and we will appear it as well along with the text.
\subsubsection{Html download}
With what codes did I downloaded the metrics I needed
\subsubsection{Social media existence}
With what codes did I downloaded the metrics I needed
\subsubsection{Readability index}
With what codes did I downloaded the metrics I needed
\subsubsection{...}
With what codes did I downloaded the metrics I needed
\paragraph{Note}
Write the sites I needed and see where to put the codes
\pagebreak  
\section{Data Analysis}
To see the other commands in action, suppose at this point of text I type
\subsection{Data loading}
What metrics I needed to download
\subsection{R}
A few words about the language i used and why
\subsection{Scripts}
With what codes did I downloaded the metrics I needed
\subsubsection{Statistical Analysis}
With what codes did I downloaded the metrics I needed
\subsubsection{Regression Model}
With what codes did I downloaded the metrics I needed
\subsubsection{Clustering}
With what codes did I downloaded the metrics I needed
\pagebreak  
\section{Conclusions}
\pagebreak  
\section{Bibliography}
\begin{thebibliography}{widest-label}

\bibitem{key1}$https://en.wikipedia.org/wiki/Fortune_500$
\bibitem{key2}$http://beta.fortune.com/fortune500$
\bibitem{key3}$http://www.tablesgenerator.com/$
\bibitem{key4}$https://www.continuum.io/downloads$
\end{thebibliography}


\newpage
\appendix
\section{\\Appendix A: Fortune 500 Companies} \label{App:AppendixA}
% the \\ insures the section title is centered below the phrase: AppendixA

\begin{table}[H]
\centering
\caption{Fortune 500 - Companies Ranked: 51 - 100}
\begin{tabular}{lll}
\hline
 & & \\
51. Intel
& 52. Humana
& 53. Disney
 \\ 
 54. Cisco Systems
& 55. Pfizer
& 56. Dow Chemical
 \\ 
57. Sysco
& 58. FedEx
& 59. Caterpillar
 \\ 
60. Lockheed Martin
& 61. N.Y. Life Insurance
& 62. Coca-Cola
 \\ 
63. HCA Holdings
& 64. Ingram Micro
& 65. Energy Transfer Equity
 \\ 
66. Tyson Foods
& 67. American Airlines Group
& 68. Delta Air Lines
 \\ 
69. Nationwide
& 70. Johnson Controls
& 71. Best Buy
 \\ 
72. Merck
& 73. Liberty Mutual I.G.
& 74. Goldman Sachs Group
 \\ 
75. Honeywell International
& 76. Massachusetts Mutual L.I.
& 77. Oracle
 \\ 
78. Morgan Stanley
& 79. Cigna
& 80. U.C. Holdings
 \\ 
81. Allstate
& 82. TIAA
& 83. INTL FCStone
 \\ 
84. CHS
& 85. American Express
& 86. Gilead Sciences
 \\ 
87. Publix Super Markets
& 88. General Dynamics
& 89. TJX
 \\ 
90. ConocoPhillips
& 91. Nike
& 92. World Fuel Services
 \\ 
93. 3M
& 94. Mondelez International
& 95. Exelon
 \\ 
96. Twenty-First Century Fox
& 97. Deere
& 98. Tesoro
 \\ 
99. Time Warner
& 100. Northwestern Mutual
 &
 \\ \hline

\end{tabular}
\end{table}

\begin{table}[H]
\centering
\caption{Fortune 500 - Companies Ranked: 101 - 150}
\begin{tabular}{lll}
\hline
 \\ 101. DuPont 
&  102. Avnet 
&  103. Macy's 
\\ 104. Enterprise Products Partners 
&  105. Travelers Cos. 
&  106. Philip Morris International 
\\ 107. Rite Aid 
&  108. Tech Data 
&  109. McDonald's 
\\ 110. Qualcomm 
&  111. Sears Holdings 
&  112. Capital One Financial 
\\ 113. EMC 
&  114. USAA 
&  115. Duke Energy 
\\ 116. Time Warner Cable 
&  117. Halliburton 
&  118. Northrop Grumman 
\\ 119. Arrow Electronics 
&  120. Raytheon 
&  121. Plains GP Holdings 
\\ 122. US Foods Holding 
&  123. AbbVie 
&  124. Centene 
\\ 125. Community Health Systems 
&  126. Alcoa 
&  127. International Paper 
\\ 128. Emerson Electric 
&  129. Union Pacific 
&  130. Amgen 
\\ 131. U.S. Bancorp 
&  132. Staples 
&  133. Danaher 
\\ 134. Whirlpool 
&  135. Aflac 
&  136. AutoNation 
\\ 137. Progressive 
&  138. Abbott Laboratories 
&  139. Dollar General 
\\ 140. Tenet Healthcare 
&  141. Eli Lilly 
&  142. Southwest Airlines 
\\ 143. Penske Automotive Group 
&  144. ManpowerGroup 
&  145. Kohl's 
\\ 146. Starbucks 
&  147. Paccar 
&  148. Cummins 
\\ 149. Altria Group 
&  150. Xerox 
 &
 \\ \hline

\end{tabular}
\end{table}

\begin{table}[H]
\centering
\caption{Fortune 500 - Companies Ranked: 151 - 200}
\begin{tabular}{lll}
\hline
 \\ 151. Kimberly-Clark 
&  152. Hartford F.S.G. 
&  153. Kraft Heinz 
\\ 154. Lear 
&  155. Fluor 
&  156. AECOM 
\\ 157. Facebook 
&  158. Jabil Circuit 
&  159. CenturyLink 
\\ 160. Supervalu 
&  161. General Mills 
&  162. Southern 
\\ 163. NextEra Energy 
&  164. Thermo Fisher Scientific 
&  165. American Electric Power 
\\ 166. PG\&E Corp. 
&  167. NGL Energy Partners 
&  168. Bristol-Myers Squibb 
\\ 169. Goodyear Tire \& Rubber 
&  170. Nucor 
&  171. PNC F.S.G. 
\\ 172. Health Net 
&  173. Micron Technology 
&  174. Colgate-Palmolive 
\\ 175. Freeport-McMoRan 
&  176. ConAgra Foods 
&  177. Gap 
\\ 178. Baker Hughes 
&  179. Bank of N.Y. Mellon C. 
&  180. Dollar Tree 
\\ 181. Whole Foods Market 
&  182. PPG Industries 
&  183. Genuine Parts 
\\ 184. Icahn Enterprises 
&  185. Performance Food Group 
&  186. Omnicom Group 
\\ 187. DISH Network 
&  188. FirstEnergy 
&  189. Monsanto 
\\ 190. AES 
&  191. CarMax 
&  192. National Oilwell Varco 
\\ 193. NRG Energy 
&  194. Western Digital 
&  195. Marriott International 
\\ 196. Office Depot 
&  197. Nordstrom 
&  198. Kinder Morgan 
\\ 199. Aramark 
&  200. DaVita HealthCare Partners 
&   
 \\ \hline
\end{tabular}
\end{table}

\begin{table}[H]
\centering
\caption{Fortune 500 - Companies Ranked: 201 - 250}
\begin{tabular}{lll}
\hline
 \\ 201. Molina Healthcare 
&  202. WellCare Health Plans 
&  203. CBS 
\\ 204. Visa 
&  205. Lincoln National 
&  206. Ecolab 
\\ 207. Kellogg 
&  208. C.H. Robinson Worldwide 
&  209. Textron 
\\ 210. Loews 
&  211. Illinois Tool Works 
&  212. Synnex 
\\ 213. Viacom 
&  214. HollyFrontier 
&  215. Land O'Lakes 
\\ 216. Devon Energy 
&  217. PBF Energy 
&  218. Yum Brands 
\\ 219. Texas Instruments 
&  220. CDW 
&  221. Waste Management 
\\ 222. Marsh \& McLennan 
&  223. Chesapeake Energy 
&  224. Parker-Hannifin 
\\ 225. Occidental Petroleum 
&  226. Guardian Life I.C.A. 
&  227. Farmers Ins. Exchange 
\\ 228. J.C. Penney 
&  229. Consolidated Edison 
&  230. Cognizant Tech. Solutions 
\\ 231. VF 
&  232. Ameriprise Financial 
&  233. Computer Sciences 
\\ 234. L Brands 
&  235. Jacobs Engineering Group 
&  236. Principal Financial 
\\ 237. Ross Stores 
&  238. Bed Bath \& Beyond 
&  239. CSX 
\\ 240. Toys R Us 
&  241. Las Vegas Sands 
&  242. Leucadia National 
\\ 243. Dominion Resources 
&  244. United States Steel 
&  245. L-3 Communications 
\\ 246. Edison International 
&  247. Entergy 
&  248. ADP 
\\ 249. First Data 
&  250. BlackRock 
&   
 \\ \hline

\end{tabular}
\end{table}

\begin{table}[H]
\centering
\caption{Fortune 500 - Companies Ranked: 251 - 300}
\begin{tabular}{lll}
\hline
\\ 251. WestRock 
&  252. Voya Financial 
&  253. Sherwin-Williams 
\\ 254. Hilton Worldwide Holdings 
&  255. R.R. Donnelley \& Sons 
&  256. Stanley Black \& Decker 
\\ 257. Xcel Energy 
&  258. Murphy USA 
&  259. CBRE Group 
\\ 260. D.R. Horton 
&  261. Estee Lauder 
&  262. Praxair 
\\ 263. Biogen 
&  264. State Street Corp. 
&  265. Unum Group 
\\ 266. Reynolds American 
&  267. Group 1 Automotive 
&  268. Henry Schein 
\\ 269. Hertz Global Holdings 
&  270. Norfolk Southern 
&  271. Reinsurance G. of America 
\\ 272. Public Service E. G. 
&  273. BB\&T Corp. 
&  274. DTE Energy 
\\ 275. Assurant 
&  276. Global Partners 
&  277. Huntsman 
\\ 278. Becton Dickinson 
&  279. Sempra Energy 
&  280. AutoZone 
\\ 281. Navistar International 
&  282. Precision Castparts 
&  283. Discover F. S. 
\\ 284. Liberty Interactive 
&  285. W.W. Grainger 
&  286. Baxter International 
\\ 287. Stryker 
&  288. Air Products \& Chemicals 
&  289. Western Refining 
\\ 290. Universal Health Services 
&  291. Owens \& Minor 
&  292. Charter Communications 
\\ 293. Advance Auto Parts 
&  294. MasterCard 
&  295. Applied Materials 
\\ 296. Eastman Chemical 
&  297. Sonic Automotive 
&  298. Ally Financial 
\\ 299. CST Brands 
&  300. eBay 
&   
 \\ \hline

\end{tabular}
\end{table}

\begin{table}[H]
\centering
\caption{Fortune 500 - Companies Ranked: 301 - 350}
\begin{tabular}{lll}
\hline
 \\ 301. Lennar 
&  302. GameStop 
&  303. Reliance Steel \& Aluminum 
\\ 304. Hormel Foods 
&  305. Celgene 
&  306. Genworth Financial 
\\ 307. PayPal Holdings 
&  308. Priceline Group 
&  309. MGM Resorts International 
\\ 310. Autoliv 
&  311. Fidelity National Financial 
&  312. Republic Services 
\\ 313. Corning 
&  314. Peter Kiewit Sons' 
&  315. Univar 
\\ 316. Mosaic 
&  317. Core-Mark Holding 
&  318. Thrivent F. for Lutherans 
\\ 319. Cameron International 
&  320. HD Supply Holdings 
&  321. Crown Holdings 
\\ 322. EOG Resources 
&  323. Veritiv 
&  324. Anadarko Petroleum 
\\ 325. Laboratory C. of A. 
&  326. Pacific Life 
&  327. News Corp. 
\\ 328. Jarden 
&  329. SunTrust Banks 
&  330. Avis Budget Group 
\\ 331. Broadcom 
&  332. American Family I. G. 
&  333. Level 3 Communications 
\\ 334. Tenneco 
&  335. United Natural Foods 
&  336. Dean Foods 
\\ 337. Campbell Soup 
&  338. Mohawk Industries 
&  339. BorgWarner 
\\ 340. PVH 
&  341. Ball 
&  342. O'Reilly Automotive 
\\ 343. Eversource Energy 
&  344. Franklin Resources 
&  345. Masco 
\\ 346. Lithia Motors 
&  347. KKR 
&  348. Oneok 
\\ 349. Newmont Mining 
&  350. PPL 
&   
 \\ \hline

\end{tabular}
\end{table}

\begin{table}[H]
\centering
\caption{Fortune 500 - Companies Ranked: 351 - 400}
\begin{tabular}{lll}
\hline
 \\ 351. SpartanNash 
&  352. Quanta Services 
&  353. XPO Logistics 
\\ 354. Ralph Lauren 
&  355. Interpublic Group 
&  356. Steel Dynamics 
\\ 357. WESCO International 
&  358. Quest Diagnostics 
&  359. Boston Scientific 
\\ 360. AGCO 
&  361. Foot Locker 
&  362. Hershey 
\\ 363. CenterPoint Energy 
&  364. Williams 
&  365. Dick's Sporting Goods 
\\ 366. Live Nation Entertainment 
&  367. Mutual of Omaha Ins. 
&  368. W.R. Berkley 
\\ 369. LKQ 
&  370. Avon Products 
&  371. Darden Restaurants 
\\ 372. Kindred Healthcare 
&  373. Weyerhaeuser 
&  374. Casey's General Stores 
\\ 375. Sealed Air 
&  376. Fifth Third Bancorp 
&  377. Dover 
\\ 378. Huntington Ingalls Industries 
&  379. Netflix 
&  380. Dillard's 
\\ 381. EMCOR Group 
&  382. Jones Financial 
&  383. AK Steel Holding 
\\ 384. UGI 
&  385. Expedia 
&  386. salesforce.com 
\\ 387. Targa Resources 
&  388. Apache 
&  389. Spirit AeroSystems H.
\\ 390. Expeditors Inter. of Washington 
&  391. Anixter International 
&  392. Fidelity N. Inf. S. 
\\ 393. Asbury Automotive Group 
&  394. Hess 
&  395. Ryder System 
\\ 396. Terex 
&  397. Coca-Cola Eur. P. 
&  398. Auto-Owners Insurance 
\\ 399. Cablevision Systems 
&  400. Symantec 
&   
 \\ \hline

\end{tabular}
\end{table}

\begin{table}[H]
\centering
\caption{Fortune 500 - Companies Ranked: 401 - 450}
\begin{tabular}{lll}
\hline
 \\ 401. Charles Schwab 
&  402. Calpine 
&  403. CMS Energy 
\\ 404. Alliance Data Systems 
&  405. JetBlue Airways 
&  406. Discovery Communic.
\\ 407. Trinity Industries 
&  408. Sanmina 
&  409. NCR 
\\ 410. FMC Technologies 
&  411. Erie Insurance Group 
&  412. Rockwell Automation 
\\ 413. Dr Pepper Snapple Group 
&  414. iHeartMedia 
&  415. Tractor Supply 
\\ 416. J.B. Hunt Transport Services 
&  417. Commercial Metals 
&  418. Owens-Illinois 
\\ 419. Harman Inter. Ind.
&  420. Baxalta 
&  421. American F. G.
\\ 422. NetApp 
&  423. Graybar Electric 
&  424. Oshkosh 
\\ 425. Ameren 
&  426. A-Mark Precious Metals 
&  427. Barnes \& Noble 
\\ 428. Dana Holding 
&  429. Constellation Brands 
&  430. LifePoint Health 
\\ 431. Zimmer Biomet H. 
&  432. Harley-Davidson 
&  433. PulteGroup 
\\ 434. Newell Brands 
&  435. Avery Dennison 
&  436. Jones Lang LaSalle 
\\ 437. WEC Energy Group 
&  438. Marathon Oil 
&  439. TravelCenters of A. 
\\ 440. United Rentals 
&  441. HRG Group 
&  442. Old Republic Inter. 
\\ 443. Windstream Holdings 
&  444. Starwood Hotels \& Resorts 
&  445. Delek US Holdings 
\\ 446. Packaging Corp. of A.
&  447. Quintiles Transnational H. 
&  448. Hanesbrands 
\\ 449. Realogy Holdings 
&  450. Mattel 
&   
 \\ \hline

\end{tabular}
\end{table}

\begin{table}[H]
\centering
\caption{Fortune 500 - Companies Ranked: 451 - 500}
\begin{tabular}{lll}
\hline
 \\ 451. Motorola Solutions 
&  452. J.M. Smucker 
&  453. Regions Financial 
\\ 454. Celanese 
&  455. Clorox 
&  456. Ingredion 
\\ 457. Genesis Healthcare 
&  458. Peabody Energy 
&  459. Alaska Air Group 
\\ 460. Seaboard 
&  461. Frontier Communic. 
&  462. Amphenol 
\\ 463. Lansing Trade Group 
&  464. SanDisk 
&  465. St. Jude Medical 
\\ 466. Wyndham Worldwide 
&  467. Kelly Services 
&  468. Western Union 
\\ 469. Envision Healthcare H. 
&  470. Visteon 
&  471. Arthur J. Gallagher 
\\ 472. Host Hotels \& Resorts 
&  473. Ashland 
&  474. Insight Enterprises 
\\ 475. Energy Future Holdings 
&  476. Markel 
&  477. Essendant 
\\ 478. CH2M Hill 
&  479. Western \& Southern F.G. 
&  480. Owens Corning 
\\ 481. S\&P Global 
&  482. Raymond James Financial 
&  483. NiSource 
\\ 484. Airgas 
&  485. ABM Industries 
&  486. Citizens F.G.
\\ 487. Booz Allen Hamilton H. 
&  488. Simon Property Group 
&  489. Domtar 
\\ 490. Rockwell Collins 
&  491. Lam Research 
&  492. Fiserv 
\\ 493. Spectra Energy 
&  494. Navient 
&  495. Big Lots 
\\ 496. Telephone \& Data Systems 
&  497. First American Financial 
&  498. NVR 
\\ 499. Cincinnati Financial 
&  500. Burlington Stores 
&
 \\ \hline

\end{tabular}
\end{table}











\end{document}
