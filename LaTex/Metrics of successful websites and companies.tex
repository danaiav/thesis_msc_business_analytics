\documentclass{article}

\usepackage{titlesec}
\setcounter{secnumdepth}{4}
\titleformat{\paragraph}
{\normalfont\normalsize\bfseries}{\theparagraph}{1em}{}
\titlespacing*{\paragraph}
{0pt}{3.25ex plus 1ex minus .2ex}{1.5ex plus .2ex}

\usepackage{hyperref}
\usepackage{float}
\usepackage{listings}
\begin{document}
\title{Metrics of successful websites and companies}
\author{Danai Avratoglou}
\date{January 2017}
\maketitle
\newpage
\tableofcontents
\newpage
\section{Introduction}
 
An on-line presence of a company was not an important factor of the overall success that the enterprise would have until a few years ago. Taking although into account the vast spread of the impact that internet has on consumers regarding their brand choices and their products this hypothesis is not valid any more. Companies are obliged by the trends to be active on-line and to maintain a website that depicts the image they want their consumers to perceive. By creating a more consumer orientated image they lead the users to create a positive idea regarding the company and this could potentially lead to bigger sales. The purpose of this paper is to understand the relationship that exists between the website of a company and its success. As success we will consider the amount of revenues of the company. Trying to comprehend this relationship a comparison will take place between the enterprises that were deemed as the more successful ones from Fortune 500  based on their revenues and a number of website metrics in order to see by performing regression models and statistical analysis which metrics do affect the most or are related the most with a company's success.
\newpage
\section{Data gathering}
The first step in order to contact this research is to find which companies we are going to examine. Since the purpose of this paper is to see if the website metrics that will be examined are influencing the success of the company it is a good idea to examine websites of some already successful companies and try to find out what they have in common. Thus as it is mentioned in the introduction we will examine the 500 companies that were ranked as the most successful ones from Fortune 500.
\subsection{Data Source - Fortune 500}
The Fortune 500 is an annual list compiled and published by Fortune magazine that ranks 500 of the largest United States corporations by total revenue for their respective fiscal years. The list includes public companies, along with privately held companies for which revenues are publicly available.\cite{key1, key2}\\ 
For the purposes of this paper we will use this list of companies and we will examine their websites in order to understand if they indeed have something in common or if their success is irrelevant with their on-line presence.\\
The first thing that we will need is a list of the names that are include in the fortune 500. The easiest way to obtain this list is from the following article:
\href{url}{http://www.zyxware.com/articles/4344/list-of-fortune-500-companies-and-their-websites}.\ The way that we will obtain the list will be explained further along in this paper.\\
In the following table one can see the 50 most successful companies that are included in the Fortune 500 during the period this paper is taking place. The rest of the list is available in the Appendix A.\ref{App:AppendixA}
\begin{table}[H]
\centering
\caption{Fortune 500 - 50 first companies}
\begin{tabular}{lll}
\hline
 \\ 1. Walmart 
&  2. Exxon Mobil 
&  3. Apple 
\\ 4. Berkshire Hathaway 
&  5. McKesson 
&  6. UnitedHealth Group 
\\ 7. CVS Health 
&  8. General Motors 
&  9. Ford Motor 
\\ 10. AT\&T 
&  11. General Electric 
&  12. AmerisourceBergen 
\\ 13. Verizon 
&  14. Chevron 
&  15. Costco 
\\ 16. Fannie Mae 
&  17. Kroger 
&  18. Amazon.com 
\\ 19. Walgreens Boots Alliance 
&  20. HP 
&  21. Cardinal Health 
\\ 22. Express Scripts Holding 
&  23. J.P. Morgan Chase 
&  24. Boeing 
\\ 25. Microsoft 
&  26. Bank of America Corp. 
&  27. Wells Fargo 
\\ 28. Home Depot 
&  29. Citigroup 
&  30. Phillips 66 
\\ 31. IBM 
&  32. Valero Energy 
&  33. Anthem 
\\ 34. Procter \& Gamble 
&  35. State Farm Insurance Cos. 
&  36. Alphabet 
\\ 37. Comcast 
&  38. Target 
&  39. Johnson \& Johnson 
\\ 40. MetLife 
&  41. Archer Daniels Midland 
&  42. Marathon Petroleum 
\\ 43. Freddie Mac 
&  44. PepsiCo 
&  45. United Technologies 
\\ 46. Aetna 
&  47. Lowe's 
&  48. UPS 
\\ 49. AIG 
&  50. Prudential Financial 
&
 \\ \hline

\end{tabular}
\end{table}

\subsection{Metrics}
Now that we have declared the companies that we are going to use we will also need to decide which metrics are we going to examine for each site. Since we cannot have access to metrics such as traffic we will have to examine metrics that are more related to how the site is structure and what exactly does the initial page of each site includes. Initially we can divide the metrics in two major categories:
\begin{itemize}
\item What we see?
\item What lays behind of what we see?
\end{itemize}
In the first category we are referring to metrics that can easily be conceived by the naked eye as well. For example the images that a website is using in its landing page. How many there are and if they are big or small.\\
The second category is not so obvious and it includes informations that usually is visible only to the web developer or the creator of the page. The information here are being given from the html code of a site. For instance we can see the actual type and size of an image, an information not visible with the naked eye.\\
Now that we have a first understanding of the two main categories that the metrics we will use are divided in, we can see in detail the metrics that will be examined in this paper:
\subsubsection{Loading time:}One aspect of a website that is crucial is the time it takes for it to load. Nowadays that the internet speed is going higher and higher most people do not have the patient to wait for a page to load. That is why we think that a metrics that should be definitely included in this research in the loading time of the initial page of a site.
\subsubsection{Number of links:}When someone is browsing through the internet in many cases they are not completely sure what exactly they are looking for and that is why in a website it would be wise to have some links that can direct the user to find what he wants.These links can either direct the user in another page of the same site, in which case the link will be characterized as an internal link. Or they can lead the user to another site, where in that case the link is characterized as an external link.For the purpose of this paper since it is not so clear which type of links are more important to a user we will examine both the internal and the external links.
\subsubsection{Social media:}Our era is marked by the social media wave that has changed our lifestyle and our daily habits. So it would be considered an overview if we didn't take under consideration the number of social media that the company chooses to participate in. Even though they can also be considered as external links of the website we will examine them separately in order to see if any particular social medium effects the company's revenues.
\subsubsection{Number, type and sizes of images:}Since the site is the first thing that a user will see for the company and there is a famous quote that says that \textit{"First impressions counts"} we should also examine how the companies decide to visualize their landing page. In other words to see how many images they include in it and more on that what sizes are those images.\\ It is completely different to see only one huge image in the landing page of a site with not many words or descriptions than to see many small images with different information. The purpose is to examine if these type of diversities between the examined websites are actually related to how they are doing profit wise.\\ Moreover a kind of information a little more complicated for a simple user to understand, but plays a very important role in many cases is the type of the image. For example some websites are using specific type of images or banners that are not compatible with all the browsers, leading the user to see some break points in the website and even stop visiting it. That is why we believe is important to review this metric as well.
\subsubsection{Content:} They say a picture worth a thousand words but that is not enough in our case. After exploring the number, sizes and type of pictures that a website is using we should also explore the number of words it is using to accompany the images and complete the outcome that a user will come across. The metrics we will use will be two. The first one will be the total words that are being used in the landing page and the other one will be the total unique words that are being used. When we are referring to unique words we mean words that are not so commonly use such as "a" or "and" and they give an air of individuality to the text. By using this metric we would try to see if the words that are being used are just as important as the actual content and if the words can make a difference.\\
Furthermore we should also take under consideration how comprehending is the text used in the websites for the users. This information can be obtained by calculating the readability index of the website and also the number of sentences that exists in a page (a metric that comes to complete the previous ones).How the readability index is being calculating will be furthered explained in another section.
\subsubsection{HTML Validation:} Moreover we will have to check the quality of the html code behind the website we are seeing. Are there any mistakes in the code for example any brackets that opened and never closed or any links that do not work. We will examine again two different metrics here the number of errors and the number of warnings. The warning are parts of the code that even though they work at the time there is a good chance to malfunction if any changes or addition are to be made to the html code. 
\subsection{Python}
After having a first look into the variables/ metrics we will use in order to contact this research we should now see how we are going to obtain all this information.\\
Since all of this information can be subtract from the html code of a company's website we should use a programming language in order to download the html pages and then to extract the specific metrics we want to examine from them.\\
For the purposes of this paper the programming language that will be used for downloading the metrics from the websites is Python. More specifically the version of Python that will be used is the 2.7 one.\\
Python is a widely used high-level programming language used for general-purpose programming, created by Guido van Rossum and first released in 1991. An interpreted language, Python has a design philosophy which emphasizes code readability (notably using white space indentation to delimit code blocks rather than curly braces or keywords), and a syntax which allows programmers to express concepts in fewer lines of code than possible in languages such as C++ or Java. \\
The language provides constructs intended to enable writing clear programs on both a small and large scale. Furthermore the way that Python allows a user to programming is common to all users which gives this language a leverage as a program build in Python can be easily understood from another user without any difficulty.\\
The environment that is going to be used is from the Anaconda package which is a free open source distribution of the Python and R programming languages for large-scale data processing, predictive analytics, and scientific computing, that aims to simplify package management and deployment. To be more precise from this package we are going to use the Jupyter Notebook.\cite{key4} \\
\subsection{Scripts}
In order to gather all the needed metrics we had to create a variety of small scripts so as to collect them. In this section we will present in detail the procedure and the actual scripts that were used in order to extract the information that later will help us contact the analysis of the relationship between those metrics and the company's status.
\subsubsection{Companies ranking, names and url}
For starters we need to create a list with the names, the ranking and the URL of the sites that we will later download and extract the necessary info from them. So as it was mentioned in the previous section the first step that needs to be done is to download and gather those informations into a data frame\footnote{a table in Python environment with rows and columns} so as to be able to use them later on.\\
The most easy way to obtain those informations is by extracting them from an already existing list. This list in our case was available in the following link:\href{url}{http://www.zyxware.com/articles/4344/list-of-fortune-500-companies-and-their-websites}.\\
The way to keep only those three informations as three different variables is by separating from the html code of this page the needed information.\\
The first step is to create 3 empty lists where we will include the informations we are going to extract. The first list will contain the rank of each site, the second one will contain the name of the company and the 3rd one will contain the actual link of the company's site:
\begin{lstlisting}[language=Python]
list_company_number =[]
list_company_name = []
list_company_website = []
\end{lstlisting}
The second step is to upload some libraries that will help us create this function but also the ones that are going to follow.
\begin{lstlisting}[language=Python]
import urllib
import urllib2
import time
import os
from bs4 import BeautifulSoup
import re
import numpy as np
import pandas as pd
import matplotlib.pyplot as plt
\end{lstlisting}
Finally the third step is to create the function that will firstly download the html code of the url at hand, secondly keep only the part of the code that we need to examine and thirdly save this part into the empty lists we created above. This function is called websites and takes as variable to work only the url of the site we need to examine: 
\begin{lstlisting}[language=Python]
def websites (url): 
    from time import time
    start = time ()
    browser = urllib2.build_opener() 
    browser.addheaders = [('User-agent', 'Mozilla/5.0')]
    response = browser.open(url)
    myHTML = response.read()
    soup = BeautifulSoup(myHTML,"lxml")    
    o = 0
    td_list =[]
    for row2 in soup.html.body.findAll('td'):
        td_list.insert(o, row2)
        o = o + 1
    a = 0
    b = 1
    c = 2
    list_numbering = 0
    for i in range (0,500):        
        num = str(td_list[a])
        company = str(td_list[b])
        site = str(td_list[c])
        c_num = re.findall('>(.+?)</td>',num)  
        c_num = str(c_num[0])
        c_name = re.findall('>(.+?)</td>',company)
        c_name = str(c_name[0])
        c_site = re.findall('">(.+?)</a>',site)
        c_site = str(c_site[0])        
        list_company_number.insert(list_numbering,c_num)
        list_company_name.insert(list_numbering,c_name)
        list_company_website.insert(list_numbering,c_site)
        a = a + 3
        b = b + 3
        c = c + 3
        list_numbering =  list_numbering + 1 
    end = time ()
    duration = round (end - start, 1)
    minutes = round (duration /60, 1)
    print 'The lists are ready in ', duration, ' seconds'
    print 'The lists are ready in ', minutes, ' minutes'
\end{lstlisting}
The steps we followed to create the following function are the following:
\paragraph{Step 1} We create a fake browser that we are going to use in order to open the page and downloaded. The reason we do that is that many sites do not allow us to download their page because they are afraid of stealing important information. Since we are not using any private information we use this method to avoid issues while trying to open the html page at hand.    
\paragraph{Step 2} We open the url and we read it while saving it in the variable "myHTML".
\paragraph{Step 3} With the help of the BeautifulSoup library we read the page as a lxml file and then for each row of this file we are looking for the "td" parts of the code where the informations we want are included.
\paragraph{Step 4} Since we need the names and the urls of all the 500 sites we created a loop from 0 to 500 where for each i we try to isolate the part of the code that contains the information that we want. Moreover even thought it seems that with this loop we calculate 501 numbers since in Python the second bracket is always open we actually count from zero to 499.
\paragraph{Step 5} We use reg expressions\footnote{A regular expression is a special sequence of characters that helps you match or find other strings or sets of strings, using a specialized syntax held in a pattern.} in order to state precisely what part of the already selected code we want to keep.
\paragraph{Step 6} We insert with a specific order the names, the ranking and the url to the corresponding lists and finally we create a text that will appear when the function is completed. Here we have also calculated the time that this function took to be completed and we will appear it as well along with the text.
\subsubsection{URL validation}
Now that we have saved in the three lists the names, the ranking and the URL of the companies that we are going to examine we first have to check whether or not those URL are valid in order to proceed with the download of the html code behind the initial web page of each one of those companies.\\
We first have to install the validators package in python so as to proceed with the validation of the url. In the command prompt window that opens when you open the Jupyter Notebook you have to write: "pip install validators" and then press enter in order for the package to be installed. This specific package currently supports python versions 2.7, 3.3, 3.4, 3.5 and PyPy.\footnote{https://validators.readthedocs.io/en/latest} 
The function that we are going to use from this package is called validators.url. This functions returns True if the url at hand is a valid url or False if it is not.\\
Below one can see that we created a loop that will run as many time as the length of the list that we created in the previous section.\\
During this loop we create a string variable where we use the URL that we have saved in the list, one at each time, and we add the "http://" prefix.\\
Now that we have create the correct way that a url is supposed to be written we will use the function of the validators package and we will create an if function that will check if the answer to a site is different from True and if it is it would add a unit in the variable nv so as to know at the end how many sites did not had valid URL.\\
Finally we ask from the programme to print the final result so as to know how many URL are not valid. The code that was created for this procedure is the following one:
\begin{lstlisting}[language=Python]
nv = 0
for num in range(len(list_company_website)):
    line = 'http://' + str(list_company_website[num])
    x = validators.url(line)    
    if x != True:
        nv = nv +1
print "The validation is complete! There were" , nv,
 "not valid pages"
\end{lstlisting}
In our case all the URL where valid so we are eligible to proceed at the next part of the code.
\subsubsection{Download sites html code}
After checking the validity of the URL that we have saved in the list the next step is to download the actual html code of the initial pages of each one of the 500 websites we want to examine.\\
As we did in the section 2.4.1 we have to create a fake browser so as the to be able to download the html code without contacting any problems. In our case we created a Mozilla browser.\\
In order for the url to be opened we must first bring it to the correct form. Initially we create again a loop and for each loop we will examine a specific url. We first replace some symbols on the string that will not be recognized if we try to open the url in this form and then we add the "http://" in the begging of each url.\\ 
Before starting downloading we create some rules for some exceptions. For example the sites 71 (Best Bay),119 (Arrow Electronics) and 465 (St. Jude Medical) in ranking create problem when we tried to download them and the code is stop working. So in order to avoid such incidents we created an exception and the python code will not even try to download these sites and in their position in the new list with the html pages that we will create a zero will be saved.The same thing would happen if an exception is thrown in the function if that we are creating. In any other case we will open the site and save this action in the variable response2.\\
After saving the action we will use the command read and we will save the html code which is essentially a long piece of text and then we will save this "text" in the list.\\
Finally we will wait for 2 seconds in order for the browser not to be suspicious from the extreme speed that we are going to open the next page. In that way we avoid having any crushing incidents on the code.\\
After completing this procedure for all the site we will have a list that in each position the html code will be held as a very large text. Except from the sites that were not able to be opened.\\
The code that was created for this procedure is the following:
\begin{lstlisting}[language=Python]
import time
browser2 = urllib2.build_opener()
browser2.addheaders = [('User-agent', 'Mozilla/5.0')]
for i in range (0,500):
    k = str(i + 1)
    lc = str(list_company_website[i])
    lc = lc.replace("'","")
    lc = lc.replace("[","")
    lc = lc.replace("]","")
    lcn = str(list_company_name[i])
    lcn = lcn.replace("'","")
    lcn = lcn.replace("[","")
    lcn = lcn.replace("]","")
    url2= 'http://' + lc
    list500_names.insert(i,lcn)
    list500_url.insert(i,lc)
    list500_num.insert(i,k)
    if i == 118 or i == 464 or i == 70:
        list500_sites.insert(i,0)  
        print ("The site " + str(i) 
        + " has NOT been downloaded!")
    else:
        try:
            response2=browser2.open(url2)
            print ("The site " + str(i) 
            + " has been downloaded!")
        except Exception:
            list500_sites.insert(i,0)
            print ("The site " + str(i) 
            + " has NOT been downloaded from exception!")           
            continue 
        myHTML2=response2.read()
        list500_sites.insert(i,myHTML2)        
        time.sleep(2)         
\end{lstlisting}
\subsubsection{Not downloadable pages}
As we explained in the previous section there are some pages that were not able to be downloaded. In order to know which are those pages and more specifically which companies sites will not be available for further exploration we created this part of the code which creates a list with the names of those companies.\\
In the previous code we used a zero in each position that we weren't able to download the site. Here we will use this information in order to gather in a new list only the parts of the list that did get a zero.\\
Here we actually created a small function that is called $"not_downloadables"$. In the first part of the code we create the function and in the second part we run the function for our lists. Finally we create a data frame with the results so as to be more easy on the eyes.\\
The code that was created is the following:
\begin{lstlisting}[language=Python]
not_d = []
not_d_n = []
num = []
def not_downloadables (list500_names,list500_sites):
    met = 0       
    for i in range(len(list500_names)):       
        if list500_sites[i] == 0:
            ct = list500_names[i]
            not_d.insert(met,ct)
            not_d_n.insert(met,str(i))
            num.insert(met,met)
            met = met + 1

not_downloadables (list500_names,list500_sites)

d = {'company' : pd.Series(not_d, index=[num]),
     'number' : pd.Series(not_d_n, index=[num])}
nd = pd.DataFrame(d)    
nd
\end{lstlisting}
In the following table we can see the sites that were not downloaded: 
\begin{table}[H]
\centering
\caption{Not downloadable sites}
\begin{tabular}{ll}
\hline
 &  \\ 
16. Fannie Mae 
& 63. HCA Holdings \\
71. Best Buy
& 91. Nike \\
98. Tesoro 
& 119. Arrow Electronics\\
136. AutoNation
& 142. Southwest Airlines \\
162. Southern 
& 165. American Electric Power\\
196. Office Depot 
& 217. PBF Energy \\
229. Consolidated Edison
& 240. Toys “R” Us \\
243. Dominion Resources 
& 276. Global Partners\\
307. PayPal Holdings 
& 327. News Corp. \\
364. Williams 
& 415. Tractor Supply\\ 
442. Old Republic International 
& 465. St. Jude Medical \\
\hline

\end{tabular}
\end{table}


\subsubsection{Content}
Now that we have downloaded the html code we should start extracting some of the variables we are going to use for the analysis in the next chapter.\\
The first thing we are going to check is how easy it is for a user to read the content of each site. In order to check this performance indicator we will need to gather 4 variables:
\begin{enumerate}
\item Number of Words
\item Number of Unique Words
\item Number of Sentences
\item Flesch score
\end{enumerate}
The three first are quite clear. The number of words are the total words that appear in the texts that the user is seeing in a website. The number of unique words are the words that are not very common and can attract the readers attention or even make it harder for him to comprehend the text. The number of sentences is related to those variables as we can understand from this is comparison to the number of total words how big or short is the sentence and as a conclusion how easy or not is the complete text.\\
Finally regarding the forth variable the flesh score is referring to the Flesh reading ease test score.In the Flesch reading-ease test, higher scores indicate material that is easier to read; lower numbers mark passages that are more difficult to read. The formula for the Flesch reading-ease score (FRES) test is:\\ \\
$206.835 - 1.015 (total words/total sentences) - 84.6 (total syllables/total words)$\\
\\
We will calculate the flesh reading test and the other three aforementioned variables with the help of an on-line checking site called: \\
"http://www.webpagefx.com/tools/read-able/".\\
This site calculates and return all these results that we need. The way to implement them in a new data frame to our python code is by dividing the html code, behind the page with the results for each site, and keeping only the actual numbers and sizes that we seek. \\
\begin{lstlisting}[language=Python]
flesch = []
sentence = []
word = []
unique_w =[]
empty =[]

import time 
for num in range(0,500):
    site = list500_sites[num]
    line = list500_url[num] 
    url_check = "http://www.webpagefx.com/tools/
    read-able/check.php?tab=Test+By+Url&uri=http://"+ line
    browser = urllib2.build_opener()
    browser.addheaders = [('User-agent', 'Mozilla/5.0')]
    if site == 0 or num == 107:
        print("Site", str(num), "is not validated from sites")
        flesch.insert(num,"n/a")
        sentence.insert(num,"n/a")
        word.insert(num,"n/a")
        unique_w.insert(num,"n/a")  
    else:
        try:
            response = browser.open(url_check)
        except Exception: 
            flesch.insert(num,"n/a")
            sentence.insert(num,"n/a")
            word.insert(num,"n/a")
            unique_w.insert(num,"n/a")
            print("Site", str(num), "is not validated from check")
            continue        
        html_r = response.read()
        check = str(html_r)       
        if check != empty:                
                soup = BeautifulSoup(check,"lxml")
                o = 0
                keyf = []
                for row in soup.html.body.findAll('tr'):
                    keyf.insert(o,row)
                    o = o + 1
                if keyf != empty:                        
                        print("Site", str(num), "is validated")
                        #Flesh measurement
                        if keyf[0] != empty:
                            readability = str(keyf[0])
                            split1 = readability.split('>')
                            readability2 = str(split1[4])
                            split2 = readability2.split('<')
                            readability3 = str(split2[0])
                            flesch.insert(num,readability3)
                        else:
                            flesch.insert(num,"n/a")
                            sentence.insert(num,"n/a")
                            word.insert(num,"n/a")
                            unique_w.insert(num,"n/a")   
                        #Number of sentences   
                        if keyf[6] != empty:
                            sentences = str(keyf[6])
                            spli1 = sentences.split('>')
                            sentences2 = str(spli1[4])
                            spli2 = sentences2.split('<')
                            sentences3 = str(spli2[0])
                            sentence.insert(num,sentences3)
                        else:
                            flesch.insert(num,"n/a")
                            sentence.insert(num,"n/a")
                            word.insert(num,"n/a")
                            unique_w.insert(num,"n/a")  
                        #Number of words
                        if keyf[7] != empty:
                            words = str(keyf[7])
                            spl1 = words.split('>')
                            words2 = str(spl1[4])
                            spl2 = words2.split('<')
                            words3 = str(spl2[0])
                            word.insert(num,words3)
                        else:
                            flesch.insert(num,"n/a")
                            sentence.insert(num,"n/a")
                            word.insert(num,"n/a")
                            unique_w.insert(num,"n/a")  
                        #No. of complex words
                        if keyf[7] != empty:
                            unique_ws = str(keyf[8])
                            sp1 = unique_ws.split('>')
                            unique_ws2 = str(sp1[4])
                            sp2 = unique_ws2.split('<')
                            unique_ws3 = str(sp2[0])
                            unique_w.insert(num,unique_ws3)
                        else:
                            flesch.insert(num,"n/a")
                            sentence.insert(num,"n/a")
                            word.insert(num,"n/a")
                            unique_w.insert(num,"n/a")  
                else:
                        print("Site", str(num), "is not validated from check 2")
                        flesch.insert(num,"n/a")
                        sentence.insert(num,"n/a")
                        word.insert(num,"n/a")
                        unique_w.insert(num,"n/a")            
    time.sleep(2)
\end{lstlisting}
\textbf{Step 1} \\We create four lists where we will save the variables that we are looking for. Then we start a loop for the 500 sites where in each loop we change the value of the 3 main variables the sites that contain the html code the url that contain the url of the company at hand each time and the url check where we save the part of the web address that remains the same while doing the on line check and we add at the end the url of the site we want to check. Also we open the browser as we have done in previous scripts as well.\\\\
\textbf{Step 2} \\Then we create an if part where we check that the variable site is not zero and that we will not examine the site 108 (Tech Data) as the code seems to have a problem in that case. In this check we put on the respective sites n/a values so as to be clear that we did not retrieve these info for them.\\\\
\textbf{Step 3} \\We open the browser link and we check whether or not we drop to any exception in which case we also put n/a values in the respective companies.\\\\
\textbf{Step 4} \\We read the link and after checking that it is not empty (this we  can check by comparing it with an empty list that we have created for this use) we use the beautiful soup package to extract the part of the code that we need.\\\\
\textbf{Step 5} \\After taking a look at the html code of the pages we find out that the parts that we need are between the following brackets <tr>...</tr> so we will extract each such bracket from the code and save them in a list. By checking the list we found which numbers of the list items we want and we saved them in the variables we have created. Always of course checking first that the specific prices exist or in any other way we again put n/a.\\\\
\textbf{Step 6} \\Now that we have the for key indicators that we will use later on the analysis we should also create a more easily comprehend variable that is related to the flesh measure variable we just created. The numbers that each site gathered mean different things. So we will try to create a correlation and build a variable called readability that will say in text how easily read or not a site is. The score can be interpreted by the following logic:
\begin{itemize}
\item Flesch measure $<$ 30 : Very Confusing
\item Flesch measure $>$ 30 : Difficult
\item Flesch measure $>$ 50 : Fairly Difficult
\item Flesch measure $>$ 60 : Standard
\item Flesch measure $>$ 70 : Fairly Easy
\item Flesch measure $>$ 80 : Easy
\item Flesch measure $>$ 90 : Very Easy
\end{itemize}
\textbf{Step 7} \\
We created a function that creates a new list where the flesh measure variable is being saved as the above descriptions based on the scores each of the site achieved.\\\\
\textbf{Step 8} \\
Now that we have all the needed informations in lists we should combine them by creating one data frame with the use of the variable company name again in order to have a common key so as to merge all the data frames that we will create in the end.
\begin{lstlisting}[language=Python]
readability = []
def readable (flesch):
    for i in range (len(flesch)):
            f_n = flesch[i]
            if f_n == "n/a":
                readability.insert(i,"n/a")                
            else:
                a = int(float(f_n))
                if a > 90:    
                    readability.insert(i,"Very easy")                    
                elif a > 80:
                    readability.insert(i,"Easy")
                elif a > 70:
                    readability.insert(i,"Fairly easy")
                elif a > 60:
                    readability.insert(i,"Standard")
                elif a > 50:
                    readability.insert(i,"Fairly difficult")
                elif a > 30:
                    readability.insert(i,"Difficult")
                else:
                    readability.insert(i,"Very Confusing")                    
    print "The function is completed!"

readable (flesch)

d1 = {'company' : pd.Series(list500_names, index=[list500_num]),
      'url' : pd.Series(list500_url, index=[list500_num]),
      'Readability' : pd.Series(readability, index=[list500_num]),
      'Flesh_Mesaure' : pd.Series(flesch,index=[list500_num]),
'Sentences' : pd.Series(sentence, index=[list500_num]),
'Words' : pd.Series(word, index=[list500_num]),
'Unique words' : pd.Series(unique_w, index=[list500_num])}
fre = pd.DataFrame(d1)    
\end{lstlisting}

\subsubsection{HTML Validation}
After downloading and saving in data frames the first batch of metrics that we will need for the analysis that we will do further along we should also check the quality of the html code.\\
Most pages on the World Wide Web are written in computer languages (such as HTML). One of the advantages of writing in a computer language is that it allows the developer to structure text, add multimedia content, and specify what appearance, or style, the result should have based on his needs.\\
As every speaking language,so the computer languages do have their own grammar, vocabulary and syntax too. Each document that is written in these computer languages is supposed to follow these rules in order to consider it well structure and written.\\
However, just as texts in a natural language can include spelling or grammar errors, documents using computer languages may (for various reasons) not be following these rules as well.\\
The process of verifying whether or not a document actually follows the rules for the language it uses is called validation, and the tool used for that is a validator. A document that passes this process with success is called valid.\\
With these concepts in mind, we can define "validation" as the process of checking a web document against the grammar (generally a DTD\footnote{DTT: Document Type definition}) it claims to be using.\\
For the purposes of this paper we will use an on-line validator site called W3C\footnote{https://validator.w3.org/} which will help as see how many errors and warnings does each site have. \\\\
\textbf{Step 1}\\
Initially we create the empty lists in which we will save the variables we will extract that will give as a clear glance on how many errors does each page has, how many warnings, how many pages weren't recognized as documents and how many pages did open during the process.\\\\
\textbf{Step 2}\\
The next step is to create a function that will do a similar job as the script we used to extract the previous variables. Initially we locate the part of the url that remains the same when the check is completed and we see the part that does change where the name of the url we want to examine should go. In order for the code to recognize that we save the result in a variable that in the end is a new url that when we open it, it will show as the page of the results for each site in each loop that we are examining.\\\\
\textbf{Step 3}\\
One difference in this script is that now the part of the code that we need is not between <tr>..</tr> but between <div>...</div> parts of the code. So we follow the same procedure as before and we locate in the list of all the divs that we create with the help of the Beatuful soup package the parts that gives us the informations we want to keep.\\\\
\textbf{Step 4}\\
One other difference is that here after the procedure ends we have counted the actual time we needed to complete this procedure. Then next step is to run the function we created.\\\\
\textbf{Step 5}\\
Finally we save the lists that we created in a new data frame where the common column remains the name of the company as it already is in the previous dataframes we have created.
\begin{lstlisting}[language=Python]
num_errors = []
num_info_warnings = []
num_non_doc = [] 
nm = []
num_open_page = []
empty = ""
 
def html_validation (list500_url,list500_names):
    from time import time # I used it to see how much time it does to run the function
    start = time ()
    for num in range(len(list500_names)):
        line = list500_url[num] 
        url_check = "https://validator.w3.org/nu/?doc=https://" + line
        browser = urllib2.build_opener()
        browser.addheaders = [('User-agent', 'Mozilla/5.0')]
        response = browser.open(url_check)
        html_check = response.read()
        html_check
        check = str(html_check)
        er = 0
        err = 0
        errr = 0
        e = False
        if check != empty:
            e = True
            soup = BeautifulSoup(check,"lxml")
            o = 0
            keyf = []
            for row in soup.html.body.findAll('div'):
                keyf.insert(o,row)
                o = o + 1                    
            if len(keyf) != 0:       
                    keyfin = str(keyf[2])                     
                    dol= re.findall('class="error"',keyfin)            
                    er = er + len(dol)
                    doll= re.findall('class="info warning"'
                                     ,keyfin)            
                    err = err + len(doll)
                    dolll= re.findall('class="non-document-error io"'
                                      ,keyfin)            
                    errr = errr + len(dolll)
        num_errors.insert(num,er)
        num_info_warnings.insert(num,err)
        num_non_doc.insert(num,errr)  
        nm.insert(num,num) 
        num_open_page.insert(num,e)
    end = time ()
    duration = round (end - start, 3)
    minutes = round (duration /60, 1)
    print 'The lists are ready in ', minutes, ' minutes'
 

html_validation (list500_url,list500_names)
 

d8 = {'company' : pd.Series(list500_names, index=[nm]),
      'The_page_opened' : pd.Series(num_open_page, index=[nm])
      ,'number_of_errors' : pd.Series(num_errors, index=[nm]),
      'number_of_warning' : pd.Series(num_info_warnings, index=[nm])
      ,'non-document-error' : pd.Series(num_non_doc, index=[nm])}
html_val = pd.DataFrame(d8)    
html_val.head(3) 
\end{lstlisting}
\subsubsection{Social media}
One other group of variables that we should include in our analysis is the social media that each company choose to use. Since this era is being characterized from the massive use of social media we can't help but wonder if some of them could actually play a crucial role in a business success.\\
The way we are going to see which social media does a company use is quite simple. We will create a list with the name of each of the  six most well known social media with the suffix ".com" and then we will check if the expression appears in the html code of each company. If an expression does exists that means that the specific company does indeed uses this specific social media and as so it appears a respective link in it's home page so as the user to have the opportunity to subscribe in it.\\
The procedure we will use is similar to the previous ones. Firstly we create the lists for the social media that we want to examine. More specifically the social media we will search for are the following:
\begin{itemize}
\item\textbf{Facebook}: a social media that lets you upload images, thoughts, songs and lets you interact with your friends
\item\textbf{Twitter}: an on-line news and social networking service where users post and interact with messages, "tweets," restricted to 140 characters
\item\textbf{Pinterest}: an internet photo sharing and publishing service that allows users to "Pin" pictures they like and upload their own recommendations to their "pinboards".
\item\textbf{YouTube}: a free video sharing website that lets people upload, view, and share videos. 
\item\textbf{Instagram}: an online photo and video sharing social networking service. It allows users to take pictures and videos, apply digital filters to them and share them to their followers. 
\item\textbf{Linkedin}: a social networking website for people in professional jobs. Users can make connections with other people they have worked with, post their work experience and skills, look for jobs, and look for workers.
\end{itemize}
The next step is to create a function that first creates the lists with the 6 elements we would search on each html code. Then after checking that the html code for each specific site has been downloaded properly (in other words is not equal to zero) we search in a loop for each of the social media at hand if it exists in the html code. If it exists we put True in the respective list.After completing this procedure we export also the time we did to run the function.\\
Now that we have the function ready we run it and afterwards we again save the results in a new data frame.
\begin{lstlisting}[language=Python]
sm_f = []
sm_t = []
sm_i = []
sm_p = []
sm_y = []
sm_l = []   
sm_nm = [] 
nm = []
sm_url = []
 
def socialmedia (list500_sites,list500_names,list500_url):
    from time import time 
    start = time ()
    for i in range(len(list500_names)):        
            myHTML = list500_sites[i]
            sm = ['facebook.com','twitter.com',
                  'instagram.com','pinterest.com',
                  'youtube.com','linkedin.com'] 
            if myHTML == 0:
                sm_nm.insert(i,list500_names[i]) 
                nm.insert(i,i)
                sm_url.insert(i,list500_url[i])
                sm_f.insert(i,'n/a')
                sm_t.insert(i,'n/a')
                sm_i.insert(i,'n/a')
                sm_p.insert(i,'n/a')
                sm_y.insert(i,'n/a')
                sm_l.insert(i,'n/a')
            else:
                for index in range(len(sm)):
                    x = sm[index]
                    social = re.findall(x,myHTML)                                
                    if (len(social) > 0):
                        if x == 'facebook.com':
                            answerf = 'TRUE'
                        if x == 'twitter.com':
                            answert = 'TRUE'
                        if x == 'instagram.com':
                            answeri = 'TRUE'
                        if x == 'pinterest.com':
                            answerp = 'TRUE'
                        if x == 'youtube.com':
                            answery = 'TRUE'
                        if x =='linkedin.com':
                            answerl = 'TRUE'                   
                    else:
                         if x == 'facebook.com':
                            answerf = 'FALSE'
                         if x == 'twitter.com':
                            answert = 'FALSE'
                         if x == 'instagram.com':
                            answeri = 'FALSE'
                         if x == 'pinterest.com':
                            answerp = 'FALSE'
                         if x == 'youtube.com':
                            answery = 'FALSE'
                         if x =='linkedin.com':
                            answerl = 'FALSE'                
                sm_nm.insert(i,list500_names[i]) 
                nm.insert(i,i)
                sm_url.insert(i,list500_url[i])
                sm_f.insert(i,answerf)
                sm_t.insert(i,answert)
                sm_i.insert(i,answeri)
                sm_p.insert(i,answerp)
                sm_y.insert(i,answery)
                sm_l.insert(i,answerl)
    end = time ()
    duration = round (end - start, 3)
    minutes = round (duration /60, 1)
    print 'The lists are completed in ', minutes, ' minutes' 
    print 'The lists are ready in ', duration, ' seconds'
 
socialmedia (list500_sites,list500_names,list500_url)

d2 = {'company' : pd.Series(sm_nm, index=[nm]),
     'facebook' : pd.Series(sm_f, index=[nm]),
      'twitter' : pd.Series(sm_t, index=[nm]),
     'instagram' : pd.Series(sm_i, index=[nm]),
      'pinterest' : pd.Series(sm_p, index=[nm]),
     'youtube' : pd.Series(sm_y, index=[nm]),
      'linkedin' : pd.Series(sm_l, index=[nm]),}
social_media = pd.DataFrame(d2) 
\end{lstlisting}
\subsubsection{Links - Internal and External}
Next step is to see how many internal, external and finally total links each html code has. The links show in how many other pages does this home page leads to. If the links are external that means as we said before that they lead in a page that is not of this website. While the internal links lead to other pages in the same website.\\
By extracting this information we want to see if it is important for the user to have the ability to browse in various pages inside the site from the home page and also if it plays any role the use of external links that give the user the opportunity to be transferred in another site that is relative of course with the one he is looking in.\\
One easy way to spot the links in an html code is the prefix that they all use which is "href". So the first step (after creating the initial lists of course and the loop we do in each script) is to locate how many times does the expression "href" appears in the code. The result of this search will give us the total links of the site.\\
Next step is to separate somehow the internal and the external ones. An easy way to do that is to find all the "href" references that are followed by "="https: ". By locating specifically those expressions we have located all the external links because unlike the internal links that do not need to be written in this form the external links since they lead to other sites they should always begin with this expression.\\
Now that we have the total links and the external links as well we can easily find the internal links with a simple subtraction.:\
\begin{center}
\textit{total links - external links = internal links}
\end{center}
Finally we put the results in the respective lists for each site and we export the time we did to run the script.\\
Now that the function is completed we run it and then we create a data frame as we have done in the previous scripts as well.
\begin{lstlisting}[language=Python]
l_nm = []
l_ex = []
l_in = []
l_t = []
nm = []
l_url = []


def links (list500_sites,list500_names,list500_url):
    from time import time    
    start = time ()
    for num in range(len(list500_names)):        
            myHTML = list500_sites[num]
            if myHTML == 0:
                l_nm.insert(num,list500_names[num])            
                l_ex.insert(num,'n/a')
                l_t.insert(num,'n/a')
                l_in.insert(num,'n/a')
                nm.insert(num,num)                
            else: 
                href = re.findall('href',myHTML)
                external = re.findall('href="https:',myHTML)
                ex = (len(external))
                alllinks = (len(href))
                internal =  (len(href) - len(external))
                l_nm.insert(num,list500_names[num])            
                l_ex.insert(num,ex)
                l_t.insert(num,alllinks)
                l_in.insert(num,internal)
                nm.insert(num,num)                
    end = time ()
    duration = round (end - start, 3)
    minutes = round (duration /60, 1)
    print 'The lists are ready in ', minutes, ' minutes'
    print 'The lists are ready in ', duration, ' seconds'
 
links (list500_sites,list500_names,list500_url)

d3 = {'company' : pd.Series(l_nm, index=[nm]),
      'external' : pd.Series(l_ex, index=[nm]),
      'internal' : pd.Series(l_in, index=[nm]),
     'total links' : pd.Series(l_t, index=[nm])}
sites_links = pd.DataFrame(d3)    
\end{lstlisting}
\subsubsection{Loading time}
One very important factor as we mentioned before is the time that the site does to be loaded and with this part of the code we will count exactly that.
\begin{lstlisting}[language=Python]
#The initial lists we will need in order 
#to calculate the loading time
lt_nm = [] 
lt_time = []
nm = []
lt_url = []
 


 
#the function that will calculate the loading time
def loadtime (list_company_website,list500_names,list500_url):
    from time import time
    browser2 = urllib2.build_opener()
    browser2.addheaders = [('User-agent', 'Mozilla/5.0')]
    for num in range(len(list500_names)):
        lc = str(list_company_website[num])        
        lc = lc.replace("'","")   
        lc = lc.replace("[","")
        lc = lc.replace("]","")
        url2 = 'http://' + lc
        if num == 118 or num == 464:
            #The site 118(119) has a problem and the whole code 
            #is stacking when I run it so we will thing of this 
            #site as a not downloadable
            lt_nm.insert(num,list500_names[num])            
            lt_time.insert(num,'n/a')
            nm.insert(num,num)
            lt_url.insert(num,list500_url[num])
            #print ("The site " + str(num) + " has NOT been loaded!")
        else:
            try:
                response2 = browser2.open(url2)
            except Exception:
                lt_time.insert(num,'n/a')
                lt_nm.insert(num,list500_names[num])  
                nm.insert(num,num)
                print ("The site " + str(num)+ " has NOT been loaded!")
                continue     
            start_time = time()
            myHTML2 = response2.read()
            end_time = time()
            response2.close()
            l_t = round(end_time-start_time, 3) 
            #in order to be more readable we rounded the time
            loadt = str(l_t)
            lt_nm.insert(num,list500_names[num])            
            lt_time.insert(num,loadt)
            nm.insert(num,num)
            lt_url.insert(num,list500_url[num])
            #print ("The site " + str(num) + " has been loaded!")
    print "The function is completed!"
 


 
#running the function for the first 25 sites
loadtime (list_company_website,list500_names,list500_url)

#creating the data frame with the loading times
d4 = {'company' : pd.Series(lt_nm, index=[nm]),
      'loading time' : pd.Series(lt_time, index=[nm])}
loading_time = pd.DataFrame(d4)    
loading_time.head(3) #we see the first 3 in the data frame
\end{lstlisting}
\subsubsection{Types and number of images}
There are many different types of images that can be used in a site. Here we will examine which ones are being preferred from the majority of the examined companies and in what amount. In other words we will count the number of each type of image and then calculate the total number of images appearing in this initial page.
\begin{lstlisting}[language=Python]
#Find out how many and what type of images each site has
#first we create the initially empty lists
p_p = []
p_d = []
p_jpg = []
p_jpeg = []
p_gif = []
p_tif = []
p_tiff = []
p_bmp = []
p_jpe = []
p_nm = []
p_tt =[]
nm = []
p_url = []
 


 
#Then we create the function that will explore 
#the html pages and search for the images
def images (list500_sites,list500_names,list500_url):
    from time import time # I used it to see 
    #how much time it does to run the function
    start = time ()
    for num in range(len(list500_names)):
            myHTML = list500_sites[num] 
            image = ['.png','.dib','.jpg','.jpeg',
                     '.bmp','.jpe','.gif','.tif','.tiff'] 
            totalnumber = 0 
            if myHTML == 0:
                p_nm.insert(num,list500_names[num])            
                p_p.insert(num,'n/a')  
                p_d.insert(num,'n/a')  
                p_jpg.insert(num,'n/a')  
                p_jpeg.insert(num,'n/a')  
                p_gif.insert(num,'n/a')  
                p_tif.insert(num,'n/a')  
                p_tiff.insert(num,'n/a')  
                p_bmp.insert(num,'n/a')  
                p_jpe.insert(num,'n/a')  
                p_tt.insert(num,'n/a')
                nm.insert(num,num)
                p_url.insert(num,list500_url[num])          
            else: 
                for index in range(len(image)):
                    x = image[index]
                    photo = re.findall(x,myHTML)
                    if x == '.png':
                        p = str (len(photo))
                    if x == '.dib':
                        d = str (len(photo))
                    if x == '.jpg':
                        jpg = str (len(photo))
                    if x == '.jpeg':
                        jpeg = str (len(photo))
                    if x == '.gif':
                        gif = str (len(photo))
                    if x == '.tif':
                        tif = str (len(photo))
                    if x == '.tiff':
                        tiff = str (len(photo))
                    if x == '.bmp':
                        bmp = str (len(photo))
                    if x == '.jpe':
                        jpe = str (len(photo))
                    totalnumber = len(photo) + totalnumber
                total = str (totalnumber)
                p_nm.insert(num,list500_names[num])            
                p_p.insert(num,p)  
                p_d.insert(num,d)  
                p_jpg.insert(num,jpg)  
                p_jpeg.insert(num,jpeg)  
                p_gif.insert(num,gif)  
                p_tif.insert(num,tif)  
                p_tiff.insert(num,tiff)  
                p_bmp.insert(num,bmp)  
                p_jpe.insert(num,jpe)  
                p_tt.insert(num,total)
                nm.insert(num,num)
                p_url.insert(num,list500_url[num])
    end = time ()
    duration = round (end - start, 3)
    minutes = round (duration /60, 1)
    print 'The lists are ready in ', minutes, ' minutes'
    print 'The lists are ready in ', duration, ' seconds'
 


 
#Then we run the function for the first 20 sites for now
images (list500_sites,list500_names,list500_url)
 

    The lists are ready in  0.1  minutes
    The lists are ready in  3.341  seconds
    


 
#Finally we create a dataframe in order to see the results of the function
d5 = {'company' : pd.Series(p_nm, index=[nm]),
      '.png' : pd.Series(p_p, index=[nm]),
      '.dib' : pd.Series(p_d, index=[nm]),
      '.jpg' : pd.Series(p_jpg, index=[nm]),
      '.jpeg' : pd.Series(p_jpeg, index=[nm]),
      '.bmp' : pd.Series(p_bmp, index=[nm]),
      '.jpe' : pd.Series(p_jpe, index=[nm]),
      '.gif' : pd.Series(p_gif, index=[nm]),
      '.tif' : pd.Series(p_tif, index=[nm]),
      '.tiff' : pd.Series(p_tiff, index=[nm]), 
      'total images' : pd.Series(p_tt, index=[nm])}
images_types = pd.DataFrame(d5)    
images_types.head(3) #we see the first 3 in the data frame
\end{lstlisting}
\subsubsection{Size of images}
Aside from the number of images another important factor is the size of each image. The following code will find the different sizes that are being used in the companies at hand and keep only the unique sizes.
\begin{lstlisting}[language=Python]
 
#Now we will find the different dimensions that each site uses
#initially we create the empty lists we will need
nm = []
s_comp = []
s_dimensions = []
s_times = []
s_tt_dif_dim = []
ht = [] #list of different heights in each case
wt = [] #list of different widths in each case
h_w = [] # combinations of height and width
dif_size = []  
un_size = [] 
s_url = []
 


 
#With the below function we will gather 
#in a variable all the different dimensions 
#and in another one all the times that each 
#dimension occures for each html code
def find_dif_sizes (list_company_website,list500_names,list500_url):
    from time import time # I used it to see how much time it does to run the function
    start = time ()
    for num in range(len(list500_names)):
            nm.insert(num,num)                  
            s_comp.insert(num,list500_names[num])
            s_url.insert(num,list500_url[num])
            myHTML = list500_sites[num] 
            if myHTML == 0:
                s_dimensions.insert(num,0)
                s_times.insert(num,0)    
            else: 
                soup = BeautifulSoup(myHTML, "lxml")
                # we create 2 local variables so as to gather the 
                #different dimensions and occurencies  of each page seperately
                s_dimensions_local = []
                s_times_local = []
                hw = 0 
                # we use it for the lists of height and width
                # find all the img in the first site html.Since in some 
                #cases either the height or the width is missing we would 
                #like to keep only the ones that have both dimensions
                for tag in soup.find_all('img'):
                    h = tag.attrs.get('height', None)
                    w = tag.attrs.get('width', None)
                    #we use if to check which ones have both 
                    if h != None:
                        if w != None:
                            ht.insert(hw,h)
                            wt.insert(hw,w)
                            hw = hw + 1                        
                hw2 = 0
                for l in range(len(ht)):
                    h_w_c = ht[l] + 'x' + wt[l]    
                    #we create a str with the form (300x300) 
                    #so as to be more easily to read later on 
                    h_w.insert(hw2,h_w_c)  
                    #we put it in a new list
                    hw2 = hw2 + 1    
                if h_w == []:#we check if there are not any dimensions available
                    nm.insert(num,num)                  
                    s_comp.insert(num,list500_names[num])
                    s_dimensions.insert(num,0)
                    s_times.insert(num,0)    
                if h_w != []:#now we continue with the cases 
                    #where the dimensions are indeed available             
                    from collections import Counter
                    hw_unique = Counter(h_w)
                    hw_unique2 = str(hw_unique) 
                    #the unique different dimensions for the specific site
                    #Due to the fact that we are talking about 
                    #a list we have to split the parts we need 
                    split1 = hw_unique2.split('{')
                    a = split1[1]
                    split2 = a.split('}')
                    b = split2[0]
                    split3 = b.split(',')
                    finalsplit = []
                    fs = []
                    z = 0
                    m = 1
                    j = 0
                    z1 = 0
                    m1 = 1
                    #each of the items in split3 has a form '300x300 : 15'
                    #and in order to create the dataframe we have 
                    #to split this form and keep the informations in different list
                    for numb in split3:                
                        oldstring = numb
                        newstring = oldstring.replace("'", "")
                        new = newstring.replace("'","")
                        string = new.replace(" ","")
                        finalstring = string.split(':')
                        #the finalstring is a list that contains the dimensions
                        #and the occurencies in order toseperate in different
                        #lists we create an additional loop
                        for xx in range(len(finalstring)):
                            ax = finalstring[xx]
                            if 'x' in ax:
                                s_dimensions_local.insert(z1,finalstring[xx])
                                z1 = z1 + 1
                            else:
                                s_times_local.insert(m1,finalstring[xx])
                                m1 = m1 + 1  
                    #Now we can add to the lists the parts we created so as
                    #to have them all gathered together             
                    s_dimensions.insert(num,s_dimensions_local)
                    s_times.insert(num,s_times_local)                
    end = time ()
    duration = round (end - start, 3)
    minutes = round (duration /60, 1)
    print 'The lists are ready in ', minutes, ' minutes'
    print 'The lists are ready in ', duration, ' seconds'
 


 
#Run the function for the first 20 sites
find_dif_sizes (list500_sites,list500_names,list500_url)
 

    The lists are ready in  1.4  minutes
    The lists are ready in  81.813  seconds
    


 
#Find the unique different image dimensions and put them on a list
def unique_dif_sizes (s_dimensions,list500_names):
    ds = 0
    for num in range(len(list500_names)):
        asw = s_dimensions[num]
        if asw != 0 :
            for s in range(len(asw)):
                ss = asw[s]
                dif_size.insert(ds,ss)
                ds = ds + 1    
    dsu = 0
    for i in dif_size:
        if i not in un_size:
            un_size.insert(dsu,i)
            dsu = dsu + 1
    print(un_size)          
 


 
#Run the function unique_dif_sizes
unique_dif_sizes (s_dimensions,list500_names)

#The lists we will need for the next function
t_f_s = []
ttf = []
nm = []
com = []
 


 
#Function in order to check whether or not each 
#company has these dimensions
def dimensions_per_company (un_size,list500_names):
    from time import time 
    # I used it to see how much time it does to run the function
    start = time ()
    #t_f_s.insert(0,un_size)
    #ttf.insert(0,t_f_s)
    for num in range(len(list500_names)): 
        #print(str(num))
        s1a = s_dimensions[num] 
        #dimensions of site num
        where = [] #empty list
        wh = 0
        haveornot = []
        for er in range (len(un_size)):
            if s1a != 0 :
                for sizea in s1a:
                    if sizea == un_size[er]:
                        where.insert(wh,str(er))
                        wh = wh +1
                        break
            if str(er) in where:
                haveornot.insert(er,True)                    
            else:
                haveornot.insert(er,False)
                    
        t_f_s.insert(num,haveornot)
        ttf.insert(num,t_f_s)
        nm.insert(num,num)
        com.insert(num,list500_names[num])
    end = time ()
    duration = round (end - start, 3)
    minutes = round (duration /60, 1)
    print 'The lists are ready in ', minutes, ' minutes'
    print 'The lists are ready in ', duration, ' seconds'
 


 
#Run the function dimensions_per_company
dimensions_per_company (un_size,list500_names)
 

    The lists are ready in  0.1  minutes
    The lists are ready in  3.699  seconds
    


 
#Create an initial dataframe where we will add the sizes later on
d6 = {'company' : pd.Series(com, index=[nm])}
sizess = pd.DataFrame(d6)    
sizess.head(3)

####################################################
#Now we want to break the variable t_f_s 
#in order to add the columns to the dataframe                  
#Finally we create the data frame with the elements we found 
def final_dimensions_dataframe (un_size,t_f_s,list500_names):
    from time import time 
    # I used it to see how much time it does to run the function
    start = time ()
    for q in range(len(un_size)):
        names = un_size[q]
        var = []
        for num in range(len(list500_names)):
            a = t_f_s[num]
            var.insert(num,a[q])
        sizess[names] = pd.Series(var, index=sizess.index) 
    end = time ()
    duration = round (end - start, 3)
    minutes = round (duration /60, 1)
    print 'The lists are ready in ', minutes, ' minutes'
    print 'The lists are ready in ', duration, ' seconds'
 


 
#Run the function final_dimensions_dataframe
final_dimensions_dataframe (un_size,t_f_s,list500_names)
 

    The lists are ready in  0.0  minutes
    The lists are ready in  0.414  seconds
    


 
sizess.tail(3)
\end{lstlisting}

\subsubsection{Fortune 500 - Metrics download}
Just in order to have some metrics that will depict the status of the companies we are going to analyse we will also download some metrics from the fortune 500 sites for each company.
\begin{lstlisting}[language=Python]
#The next step is to take some informations from the fortune 500 site for each company
#In order to achieve that we should open the pages for each one of the sites seperately
#Since there is a pattern in the way the pages are named it shouldn't be difficult
#Firstly we should create the pattern with which we will download the pages
#By running the code we can see that the names of each comany are not 
#written exactly as we have saved them
#So we do need to alter the names first in order for the below function to run
 


 
#creating a new list with alterations in order for the names
#to match the ones that fortune 500 uses so that we can download the html page
list_company_name_new = []
for num in range (0,500):
    cn = list_company_name[num]
    cn = cn.replace(" ", "-")
    cn = cn.replace("&", "")
    cn = cn.replace("’", "")
    cn = cn.replace(".", "-")
    cn = cn.replace("amp;", "")    
    company = cn.lower()
    list_company_name_new.insert(num,cn)
 


 
fortune_pages = []
def fortune500 (list_company_name_new):
    from time import time # I used it to see how much time it does to run the function
    start = time ()
    for num3 in range (0,500):
        i = str (num3 +1)    
        companyname =  list_company_name_new[num3]
        browser = urllib2.build_opener() 
        #because i work from different computers with different 
        #pyhton version some commands are not recognizable in each version
        browser.addheaders = [('User-agent', 'Mozilla/5.0')]
        site_fortune = "http://beta.fortune.com/fortune500/"+companyname+"-"+ i    
        page_fortune = browser.open(site_fortune)
        html_fortune = page_fortune.read()    
        #print("fortune page for company: ", list_company_name_new[num3],i)
        fortune_pages.insert(num3, html_fortune)
    end = time ()
    duration = round (end - start, 3)
    minutes = round (duration /60, 1)
    print 'The lists are ready in ', minutes, ' minutes'
    print 'The lists are ready in ', duration, ' seconds'
 


 
#Run the function we created
fortune500 (list_company_name_new)
 

    The lists are ready in  21.6  minutes
    The lists are ready in  1295.917  seconds
    


 
#Now that we have opened the url we are going to extract 
#some informations that we need from them
#In order to do that initially we have to create 
#the variables we will need
keyf =[]
per =[]
rev_dol = []
rev_per = []
prof_dol = []
prof_per = []
assets_dol = []
assets_per = []
tse_dol = []
tse_per = []
mar_dol = []
mar_per = []
market = []
nm = []
ln = []
urln = []
empty = []
 


 
def fortune_metrics (list_company_name,list_company_website):
    x = 0
    for n in range (0,500):   #we put 25 for testing
        nm.insert(x,x)
        ln.insert(x,list_company_name[n])
        urln.insert(x,list_company_website[n])
        files = fortune_pages[x]
        soup = BeautifulSoup(files,"lxml")
        o=0
        for row in soup.html.body.findAll('tbody'):
            keyf.insert(o,row)
            o=o+1
        keyfin = keyf[0] 
        #the elements we need is in the first tbody of the code
        data = keyfin.findAll('td')

        one = str(data[0]) 
        # revenue
        two = str(data[1]) 
        # revenue in dollars we need to extract this
        revdol= re.findall('>\$(.+?)</td>',two) 
        #we keep only the numbers
        if revdol[0] != empty:
            w = revdol[0]
            a = w.replace("[", "")
            r = a.replace("]","")
            rev_dol.insert(x,r)
        else:
            rev_dol.insert(x,'not available')
        tria = str(data[2])
        # revenue in percentage we need to extract this as well
        revper= re.findall('>(.+?)%</td>',tria) 
        #we keep only the numbers
        if revper != empty:    
            w = revper[0]
            a = w.replace("[", "")
            r1 = a.replace("]","")    
            rev_per.insert(x,r1) 
        else:
            rev_per.insert(x,'not available')
        four = str(data[3])   # profit     
        five = str(data[4])   
        # profit in dollars we need to extract this   
        profdol= re.findall('>\$(.+?)</td>',five) 
        #we keep only the numbers
        if profdol != empty:
            w = profdol[0]
            a = w.replace("[", "")
            p = a.replace("]","")
            prof_dol.insert(x,p)
        else:
            prof_dol.insert(x,'not available')
        six = str(data[5])    
        # profit in percentage we need to extract this as well   
        profper = re.findall('>(.+?)%</td>',six) 
        #we keep only the numbers
        if profper != empty:
            w = profper[0]
            a = w.replace("[", "")
            p1 = a.replace("]","")    
            prof_per.insert(x,p1)
        else:
            prof_per.insert(x,'not available')
        seven = str(data[6]) #assets
        eight = str(data[7]) #assets in dollars we need to extract this
        assetsdol= re.findall('>\$(.+?)</td>',eight) 
        #we keep only the numbers
        if assetsdol != empty:
            w = assetsdol[0]
            a = w.replace("[", "")
            ass = a.replace("]","")
            assets_dol.insert(x,ass)
        else:
            assets_dol.insert(x,'not available')
        ten = str(data[9]) #Total Stockholder Equity ($M)    
        eleven = str(data[10]) 
        #Total Stockholder Equity ($M) in dollars we need to extract this
        tsedol= re.findall('>\$(.+?)</td>',eleven) 
        #we keep only the numbers
        if tsedol != empty:
            w = tsedol[0]
            a = w.replace("[", "")
            ts = a.replace("]","")
            tse_dol.insert(x,ts)
        else:
            tse_dol.insert(x,'not available')
        thirteen = str(data[12]) # market value
        fourteen = str(data[13]) 
        # market value in dollars we need to extract this
        mardol= re.findall('>\$(.+?)</td>',fourteen) 
        #we keep only the numbers
        if mardol != empty:
            w = mardol[0]
            a = w.replace("[", "")
            mar = a.replace("]","")
            mar_dol.insert(x,mar)
        else:
            mar_dol.insert(x,'not available')
        x = x + 1
    print "The function is complete!"
 


 
fortune_metrics (list_company_name,list_company_website)
 

    The function is complete!
    



d9 = {'company' : pd.Series(ln, index=[nm]),
      'Revenues $' : pd.Series(rev_dol, index=[nm]),
      'Revenues %' : pd.Series(rev_per, index=[nm]),
      'Assets $' : pd.Series(assets_dol, index=[nm]),
      'Total Stockholder Equity $' : pd.Series(tse_dol, index=[nm]),
      'Market value $' : pd.Series(mar_dol, index=[nm])}
fort500 = pd.DataFrame(d9)    
fort500.head(3)
\end{lstlisting}
\subsubsection{Merge data-frames and extract final csv file}
The final step of the code is to create a final data frame which will include all the information gathered and then create a csv file that will be used for the further analysis that will be take place with the use of the program language R.
\begin{lstlisting}[language=Python]
fort500.merge(html_val, left_on='company', right_on='company',
how='outer')
fort500.head(3)
result = pd.merge(fort500, html_val, how='inner', on=['company', 'company'])
result2 = pd.merge(social_media, fre, how='inner', on=['company', 'company'])
result3 = pd.merge(wordss, sizess, how='inner', on=['company', 'company'])
result4 = pd.merge(images_types, loading_time, how='inner',
 on=['company', 'company'])
result5 = pd.merge(result, sites_links, how='inner', on=['company', 'company'])
result6 = pd.merge(result5, result2, how='inner', on=['company', 'company'])
result7 = pd.merge(result6, result3, how='inner', on=['company', 'company'])
final3 = pd.merge(result7, result4, how='inner', on=['company', 'company'])
final3.head(3)
final3.to_csv('total_500.csv', sep=';')

data500 = pd.read_csv("total_500.csv", sep=';') 
data500.head(3)
\end{lstlisting}
\newpage  
\section{Data Analysis}
Now that we have gathered all the information needed the next step is to proceed to their analysis. For the purposes of this assignment we will use the programming language R to perform this analysis.
\subsection{R and R-Studio}
R is a programming language and software environment for statistical computing and graphics supported by the R Foundation for Statistical Computing. The R language is widely used among statisticians and data miners for developing statistical software and data analysis.\\ 
Furthermore R is freely available under the GNU General Public License, and pre-compiled binary versions are provided for various operating systems. While R has a command line interface, there are several graphical front-ends available.\\
For the purpose of this paper we will use the R-Studio which  is a free and open-source integrated development environment (IDE) for R.
\subsection{Scripts}
In this chapter we are going to present the scripts that we used throughout the analysis and we will explain elaborately how we reached to the conclusions. The analysis that we are going to perform will be explained by the following order as we performed it:
\begin{enumerate}
\item Data cleansing
\item Variable analysis and correlation
\begin{enumerate}
\item Fortune variables correlation
\item Social media analysis and correlation
\item Links analysis and correlations
\item Words analysis and correlation
\item HTML validation variables analysis and correlation
\item Image types analysis and correlation
\end{enumerate}
\item Data manipulation
\begin{enumerate}
\item Image sizes variables reconstruction
\item Remove variables
\end{enumerate}
\item Logistic Regression
\begin{enumerate}
\item Training and test set creation
\item Null and Full model creation
\item Lasso Method
\item Both Method
\item Predictions and comparison of models
\end{enumerate}
\item Comparisons and other methods
\begin{enumerate}
\item Correlation testing
\item Clustering testing
\item Final testing
\item Final model
\end{enumerate}
\end{enumerate}
\subsubsection{Data cleansing}
The first step in order to perform the analysis is to upload the final csv we have created and then change any miss fitted value or type of variable so as to be easier for us to examine them.

\subsubsection{Variable analysis and correlation}
Now that the variables are cleansed we can perform the first step of the analysis. We should see how the prices are distributed and get a first look of what we are examining.
\paragraph{Fortune variables correlation}
The first step is to determine the variable with which we will compare all the other ones. Obviously this variable will be one of the metrics we downloaded from the fortune 500 site so as to have an actual variable that shows the status of the company and compare the rest of the variables with it. 
\paragraph{Social media analysis and correlation}
Now that we have reach to the conclusion that the metric we will use from the Fortune 500 is the Revenues we will proceed with analysing the social media variables to see how they are distributed and then to see how they are correlated with the chosen variable.
\paragraph{Links analysis and correlations}
Next step is to analyse the distribution of the internal, external and total links and of course their correlation with each other and with the Revenues.
\paragraph{Words analysis and correlation}
Continuing with the analysis of the unique words and the total words with correlation to the readability index.
\paragraph{HTML validation variables analysis and correlation}
During the extraction of the html validation variables we created 4 different metrics: the number of errors, the number of warnings the non document error and the page not opened one. In this chapter we are going to analyse the distributions of those variables and also their correlation with the Revenues.
\paragraph{Image types analysis and correlation}
Last but not least we should analyse the distribution of the different types of images that are being used on the websites that we are examining and also compare the total images with the Revenues to see the correlation.
\subsubsection{Data manipulation}
In this chapter we are going to change some of the variables that we have created in order to make them more easy to be analysed and explained in relationship to the Revenues. 
\paragraph{Image sizes variables reconstruction}
After implementing the python code for the size images (2.4.10) we created almost 700 different variables. This number is almost prohibited when it comes to regression models especially when the actual number of the examined cases are a lot fewer (500). That is why we decide to group this 700 variables to 5 new ones that we will be able to incorporate to this analysis.
\paragraph{Remove variables}
Now that we have created and analysed all the needed variables we can subtract from the data frame the variables that we have decide not to include in the regression analysis that will follow.
\subsubsection{Logistic Regression}
In this chapter we will perform logistic regression to the chosen data frame in order to determine the variables that affect most the price of the Revenues of a company
\paragraph{Training and test set creation}
Before we begin with the regression we have to divide the data frame to 2 data frames so as to be able to test the model we will create. We will make a training set and a test set.
\paragraph{Null and Full model creation}
The first step is by using the training set to create a null model (that includes only the Revenues) and a full model (that includes all the variables)
\paragraph{Lasso Method}
The next step is to implement the Lasso method
\paragraph{Both Method}
We will continue by applying the both method of the logistic regression in order to see if the variables that will be kept are the same or less than the lasso method.
\paragraph{Predictions and comparison of models}
Now we will make the predictions using the test set in order to compare how well do the model that we created work in practise.
\subsubsection{Comparisons and other methods}
Now that we have a first idea of the metrics that are considered more important it would be wise to use other methods to double check the results.
\paragraph{Correlation testing}
By seeing the correlations of the chosen variables we could create models by subtracting some of them to see if the results will be improved.
\paragraph{Clustering testing}
We will use the clustering method to see how the variables are being group together. And which variables play the most important role.
\paragraph{Final testing}
Now we will compare the results to see if we can test any other theory so as to be sure about the final results.
\paragraph{Final model}
After completing the analysis we can conclude that the most important variable of the websites that can play a crucial role to the actual revenues of a company are the images. More specifically the sizes and the type of the images along with the external links.
\pagebreak  
\section{Conclusions}
This conclusion can be easily explained from the fact that the eye is drowned to specific stimulations. There are many studies that shows that where the eye is going in the first seconds that a user is visiting a page can be crucial. So the right type of image with a correct structure and use of specific image sizes can make a site more likeable and friendly to the users.\\
Example - Testing of where does the eye drops first\\
Example - Images and the importance to the psychology\\
Example - Type of images more easily and rapid loading\\
Example - Information that the user is looking for not confined only inside the same website\\
\pagebreak  
\section{Bibliography}
\begin{thebibliography}{widest-label}

\bibitem{key1}$https://en.wikipedia.org/wiki/Fortune_500$
\bibitem{key2}$http://beta.fortune.com/fortune500$
\bibitem{key3}$http://www.tablesgenerator.com/$
\bibitem{key4}$https://www.continuum.io/downloads$
\bibitem{key5}$https://validator.w3.org/$
\bibitem{key6}Rui Miguel Forte,Mastering Predictive Analytics with R,Packt Publishing Ltd.,June 2015


\end{thebibliography}

\newpage
\appendix
\section{Appendix A: Fortune 500 Companies} 
% the \\ insures the section title is centered below the phrase: AppendixA
\begin{table}[H]
\centering
\caption{Fortune 500 - Companies Ranked: 51 - 100}
\begin{tabular}{lll}
\hline
 & & \\
51. Intel
& 52. Humana
& 53. Disney
 \\ 
 54. Cisco Systems
& 55. Pfizer
& 56. Dow Chemical
 \\ 
57. Sysco
& 58. FedEx
& 59. Caterpillar
 \\ 
60. Lockheed Martin
& 61. N.Y. Life Insurance
& 62. Coca-Cola
 \\ 
63. HCA Holdings
& 64. Ingram Micro
& 65. Energy Transfer Equity
 \\ 
66. Tyson Foods
& 67. American Airlines Group
& 68. Delta Air Lines
 \\ 
69. Nationwide
& 70. Johnson Controls
& 71. Best Buy
 \\ 
72. Merck
& 73. Liberty Mutual I.G.
& 74. Goldman Sachs Group
 \\ 
75. Honeywell International
& 76. Massachusetts Mutual L.I.
& 77. Oracle
 \\ 
78. Morgan Stanley
& 79. Cigna
& 80. U.C. Holdings
 \\ 
81. Allstate
& 82. TIAA
& 83. INTL FCStone
 \\ 
84. CHS
& 85. American Express
& 86. Gilead Sciences
 \\ 
87. Publix Super Markets
& 88. General Dynamics
& 89. TJX
 \\ 
90. ConocoPhillips
& 91. Nike
& 92. World Fuel Services
 \\ 
93. 3M
& 94. Mondelez International
& 95. Exelon
 \\ 
96. Twenty-First Century Fox
& 97. Deere
& 98. Tesoro
 \\ 
99. Time Warner
& 100. Northwestern Mutual
 &
 \\ \hline

\end{tabular}
\end{table}

\begin{table}[H]
\centering
\caption{Fortune 500 - Companies Ranked: 101 - 150}
\begin{tabular}{lll}
\hline
 \\ 101. DuPont 
&  102. Avnet 
&  103. Macy's 
\\ 104. Enterprise Products Partners 
&  105. Travelers Cos. 
&  106. Philip Morris International 
\\ 107. Rite Aid 
&  108. Tech Data 
&  109. McDonald's 
\\ 110. Qualcomm 
&  111. Sears Holdings 
&  112. Capital One Financial 
\\ 113. EMC 
&  114. USAA 
&  115. Duke Energy 
\\ 116. Time Warner Cable 
&  117. Halliburton 
&  118. Northrop Grumman 
\\ 119. Arrow Electronics 
&  120. Raytheon 
&  121. Plains GP Holdings 
\\ 122. US Foods Holding 
&  123. AbbVie 
&  124. Centene 
\\ 125. Community Health Systems 
&  126. Alcoa 
&  127. International Paper 
\\ 128. Emerson Electric 
&  129. Union Pacific 
&  130. Amgen 
\\ 131. U.S. Bancorp 
&  132. Staples 
&  133. Danaher 
\\ 134. Whirlpool 
&  135. Aflac 
&  136. AutoNation 
\\ 137. Progressive 
&  138. Abbott Laboratories 
&  139. Dollar General 
\\ 140. Tenet Healthcare 
&  141. Eli Lilly 
&  142. Southwest Airlines 
\\ 143. Penske Automotive Group 
&  144. ManpowerGroup 
&  145. Kohl's 
\\ 146. Starbucks 
&  147. Paccar 
&  148. Cummins 
\\ 149. Altria Group 
&  150. Xerox 
 &
 \\ \hline

\end{tabular}
\end{table}

\begin{table}[H]
\centering
\caption{Fortune 500 - Companies Ranked: 151 - 200}
\begin{tabular}{lll}
\hline
 \\ 151. Kimberly-Clark 
&  152. Hartford F.S.G. 
&  153. Kraft Heinz 
\\ 154. Lear 
&  155. Fluor 
&  156. AECOM 
\\ 157. Facebook 
&  158. Jabil Circuit 
&  159. CenturyLink 
\\ 160. Supervalu 
&  161. General Mills 
&  162. Southern 
\\ 163. NextEra Energy 
&  164. Thermo Fisher Scientific 
&  165. American Electric Power 
\\ 166. PG\&E Corp. 
&  167. NGL Energy Partners 
&  168. Bristol-Myers Squibb 
\\ 169. Goodyear Tire \& Rubber 
&  170. Nucor 
&  171. PNC F.S.G. 
\\ 172. Health Net 
&  173. Micron Technology 
&  174. Colgate-Palmolive 
\\ 175. Freeport-McMoRan 
&  176. ConAgra Foods 
&  177. Gap 
\\ 178. Baker Hughes 
&  179. Bank of N.Y. Mellon C. 
&  180. Dollar Tree 
\\ 181. Whole Foods Market 
&  182. PPG Industries 
&  183. Genuine Parts 
\\ 184. Icahn Enterprises 
&  185. Performance Food Group 
&  186. Omnicom Group 
\\ 187. DISH Network 
&  188. FirstEnergy 
&  189. Monsanto 
\\ 190. AES 
&  191. CarMax 
&  192. National Oilwell Varco 
\\ 193. NRG Energy 
&  194. Western Digital 
&  195. Marriott International 
\\ 196. Office Depot 
&  197. Nordstrom 
&  198. Kinder Morgan 
\\ 199. Aramark 
&  200. DaVita HealthCare Partners 
&   
 \\ \hline
\end{tabular}
\end{table}

\begin{table}[H]
\centering
\caption{Fortune 500 - Companies Ranked: 201 - 250}
\begin{tabular}{lll}
\hline
 \\ 201. Molina Healthcare 
&  202. WellCare Health Plans 
&  203. CBS 
\\ 204. Visa 
&  205. Lincoln National 
&  206. Ecolab 
\\ 207. Kellogg 
&  208. C.H. Robinson Worldwide 
&  209. Textron 
\\ 210. Loews 
&  211. Illinois Tool Works 
&  212. Synnex 
\\ 213. Viacom 
&  214. HollyFrontier 
&  215. Land O'Lakes 
\\ 216. Devon Energy 
&  217. PBF Energy 
&  218. Yum Brands 
\\ 219. Texas Instruments 
&  220. CDW 
&  221. Waste Management 
\\ 222. Marsh \& McLennan 
&  223. Chesapeake Energy 
&  224. Parker-Hannifin 
\\ 225. Occidental Petroleum 
&  226. Guardian Life I.C.A. 
&  227. Farmers Ins. Exchange 
\\ 228. J.C. Penney 
&  229. Consolidated Edison 
&  230. Cognizant Tech. Solutions 
\\ 231. VF 
&  232. Ameriprise Financial 
&  233. Computer Sciences 
\\ 234. L Brands 
&  235. Jacobs Engineering Group 
&  236. Principal Financial 
\\ 237. Ross Stores 
&  238. Bed Bath \& Beyond 
&  239. CSX 
\\ 240. Toys R Us 
&  241. Las Vegas Sands 
&  242. Leucadia National 
\\ 243. Dominion Resources 
&  244. United States Steel 
&  245. L-3 Communications 
\\ 246. Edison International 
&  247. Entergy 
&  248. ADP 
\\ 249. First Data 
&  250. BlackRock 
&   
 \\ \hline

\end{tabular}
\end{table}

\begin{table}[H]
\centering
\caption{Fortune 500 - Companies Ranked: 251 - 300}
\begin{tabular}{lll}
\hline
\\ 251. WestRock 
&  252. Voya Financial 
&  253. Sherwin-Williams 
\\ 254. Hilton Worldwide Holdings 
&  255. R.R. Donnelley \& Sons 
&  256. Stanley Black \& Decker 
\\ 257. Xcel Energy 
&  258. Murphy USA 
&  259. CBRE Group 
\\ 260. D.R. Horton 
&  261. Estee Lauder 
&  262. Praxair 
\\ 263. Biogen 
&  264. State Street Corp. 
&  265. Unum Group 
\\ 266. Reynolds American 
&  267. Group 1 Automotive 
&  268. Henry Schein 
\\ 269. Hertz Global Holdings 
&  270. Norfolk Southern 
&  271. Reinsurance G. of America 
\\ 272. Public Service E. G. 
&  273. BB\&T Corp. 
&  274. DTE Energy 
\\ 275. Assurant 
&  276. Global Partners 
&  277. Huntsman 
\\ 278. Becton Dickinson 
&  279. Sempra Energy 
&  280. AutoZone 
\\ 281. Navistar International 
&  282. Precision Castparts 
&  283. Discover F. S. 
\\ 284. Liberty Interactive 
&  285. W.W. Grainger 
&  286. Baxter International 
\\ 287. Stryker 
&  288. Air Products \& Chemicals 
&  289. Western Refining 
\\ 290. Universal Health Services 
&  291. Owens \& Minor 
&  292. Charter Communications 
\\ 293. Advance Auto Parts 
&  294. MasterCard 
&  295. Applied Materials 
\\ 296. Eastman Chemical 
&  297. Sonic Automotive 
&  298. Ally Financial 
\\ 299. CST Brands 
&  300. eBay 
&   
 \\ \hline

\end{tabular}
\end{table}

\begin{table}[H]
\centering
\caption{Fortune 500 - Companies Ranked: 301 - 350}
\begin{tabular}{lll}
\hline
 \\ 301. Lennar 
&  302. GameStop 
&  303. Reliance Steel \& Aluminum 
\\ 304. Hormel Foods 
&  305. Celgene 
&  306. Genworth Financial 
\\ 307. PayPal Holdings 
&  308. Priceline Group 
&  309. MGM Resorts International 
\\ 310. Autoliv 
&  311. Fidelity National Financial 
&  312. Republic Services 
\\ 313. Corning 
&  314. Peter Kiewit Sons' 
&  315. Univar 
\\ 316. Mosaic 
&  317. Core-Mark Holding 
&  318. Thrivent F. for Lutherans 
\\ 319. Cameron International 
&  320. HD Supply Holdings 
&  321. Crown Holdings 
\\ 322. EOG Resources 
&  323. Veritiv 
&  324. Anadarko Petroleum 
\\ 325. Laboratory C. of A. 
&  326. Pacific Life 
&  327. News Corp. 
\\ 328. Jarden 
&  329. SunTrust Banks 
&  330. Avis Budget Group 
\\ 331. Broadcom 
&  332. American Family I. G. 
&  333. Level 3 Communications 
\\ 334. Tenneco 
&  335. United Natural Foods 
&  336. Dean Foods 
\\ 337. Campbell Soup 
&  338. Mohawk Industries 
&  339. BorgWarner 
\\ 340. PVH 
&  341. Ball 
&  342. O'Reilly Automotive 
\\ 343. Eversource Energy 
&  344. Franklin Resources 
&  345. Masco 
\\ 346. Lithia Motors 
&  347. KKR 
&  348. Oneok 
\\ 349. Newmont Mining 
&  350. PPL 
&   
 \\ \hline

\end{tabular}
\end{table}

\begin{table}[H]
\centering
\caption{Fortune 500 - Companies Ranked: 351 - 400}
\begin{tabular}{lll}
\hline
 \\ 351. SpartanNash 
&  352. Quanta Services 
&  353. XPO Logistics 
\\ 354. Ralph Lauren 
&  355. Interpublic Group 
&  356. Steel Dynamics 
\\ 357. WESCO International 
&  358. Quest Diagnostics 
&  359. Boston Scientific 
\\ 360. AGCO 
&  361. Foot Locker 
&  362. Hershey 
\\ 363. CenterPoint Energy 
&  364. Williams 
&  365. Dick's Sporting Goods 
\\ 366. Live Nation Entertainment 
&  367. Mutual of Omaha Ins. 
&  368. W.R. Berkley 
\\ 369. LKQ 
&  370. Avon Products 
&  371. Darden Restaurants 
\\ 372. Kindred Healthcare 
&  373. Weyerhaeuser 
&  374. Casey's General Stores 
\\ 375. Sealed Air 
&  376. Fifth Third Bancorp 
&  377. Dover 
\\ 378. Huntington Ingalls Industries 
&  379. Netflix 
&  380. Dillard's 
\\ 381. EMCOR Group 
&  382. Jones Financial 
&  383. AK Steel Holding 
\\ 384. UGI 
&  385. Expedia 
&  386. salesforce.com 
\\ 387. Targa Resources 
&  388. Apache 
&  389. Spirit AeroSystems H.
\\ 390. Expeditors Inter. of Washington 
&  391. Anixter International 
&  392. Fidelity N. Inf. S. 
\\ 393. Asbury Automotive Group 
&  394. Hess 
&  395. Ryder System 
\\ 396. Terex 
&  397. Coca-Cola Eur. P. 
&  398. Auto-Owners Insurance 
\\ 399. Cablevision Systems 
&  400. Symantec 
&   
 \\ \hline

\end{tabular}
\end{table}

\begin{table}[H]
\centering
\caption{Fortune 500 - Companies Ranked: 401 - 450}
\begin{tabular}{lll}
\hline
 \\ 401. Charles Schwab 
&  402. Calpine 
&  403. CMS Energy 
\\ 404. Alliance Data Systems 
&  405. JetBlue Airways 
&  406. Discovery Communic.
\\ 407. Trinity Industries 
&  408. Sanmina 
&  409. NCR 
\\ 410. FMC Technologies 
&  411. Erie Insurance Group 
&  412. Rockwell Automation 
\\ 413. Dr Pepper Snapple Group 
&  414. iHeartMedia 
&  415. Tractor Supply 
\\ 416. J.B. Hunt Transport Services 
&  417. Commercial Metals 
&  418. Owens-Illinois 
\\ 419. Harman Inter. Ind.
&  420. Baxalta 
&  421. American F. G.
\\ 422. NetApp 
&  423. Graybar Electric 
&  424. Oshkosh 
\\ 425. Ameren 
&  426. A-Mark Precious Metals 
&  427. Barnes \& Noble 
\\ 428. Dana Holding 
&  429. Constellation Brands 
&  430. LifePoint Health 
\\ 431. Zimmer Biomet H. 
&  432. Harley-Davidson 
&  433. PulteGroup 
\\ 434. Newell Brands 
&  435. Avery Dennison 
&  436. Jones Lang LaSalle 
\\ 437. WEC Energy Group 
&  438. Marathon Oil 
&  439. TravelCenters of A. 
\\ 440. United Rentals 
&  441. HRG Group 
&  442. Old Republic Inter. 
\\ 443. Windstream Holdings 
&  444. Starwood Hotels \& Resorts 
&  445. Delek US Holdings 
\\ 446. Packaging Corp. of A.
&  447. Quintiles Transnational H. 
&  448. Hanesbrands 
\\ 449. Realogy Holdings 
&  450. Mattel 
&   
 \\ \hline

\end{tabular}
\end{table}

\begin{table}[H]
\centering
\caption{Fortune 500 - Companies Ranked: 451 - 500}
\begin{tabular}{lll}
\hline
 \\ 451. Motorola Solutions 
&  452. J.M. Smucker 
&  453. Regions Financial 
\\ 454. Celanese 
&  455. Clorox 
&  456. Ingredion 
\\ 457. Genesis Healthcare 
&  458. Peabody Energy 
&  459. Alaska Air Group 
\\ 460. Seaboard 
&  461. Frontier Communic. 
&  462. Amphenol 
\\ 463. Lansing Trade Group 
&  464. SanDisk 
&  465. St. Jude Medical 
\\ 466. Wyndham Worldwide 
&  467. Kelly Services 
&  468. Western Union 
\\ 469. Envision Healthcare H. 
&  470. Visteon 
&  471. Arthur J. Gallagher 
\\ 472. Host Hotels \& Resorts 
&  473. Ashland 
&  474. Insight Enterprises 
\\ 475. Energy Future Holdings 
&  476. Markel 
&  477. Essendant 
\\ 478. CH2M Hill 
&  479. Western \& Southern F.G. 
&  480. Owens Corning 
\\ 481. S\&P Global 
&  482. Raymond James Financial 
&  483. NiSource 
\\ 484. Airgas 
&  485. ABM Industries 
&  486. Citizens F.G.
\\ 487. Booz Allen Hamilton H. 
&  488. Simon Property Group 
&  489. Domtar 
\\ 490. Rockwell Collins 
&  491. Lam Research 
&  492. Fiserv 
\\ 493. Spectra Energy 
&  494. Navient 
&  495. Big Lots 
\\ 496. Telephone \& Data Systems 
&  497. First American Financial 
&  498. NVR 
\\ 499. Cincinnati Financial 
&  500. Burlington Stores 
&
 \\ \hline

\end{tabular}
\end{table}











\end{document}
